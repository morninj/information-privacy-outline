\section{Privacy and Government Records and Databases}

\subsection{Public Access to Government Records}

\subsubsection{Public Records and Court Records}

\begin{enumerate}
    \item Court records are public in all states.\footnote{Casebook p. 636.}
\end{enumerate}

\paragraph{Pseuodonymous Civil Litigation: \emph{Doe v. Shakur}}

In civil suits, plaintiffs sometimes cannot remain anonymous. However, judges 
have discretion.

\begin{enumerate}
    \item Can a victim of sexual assault bring civil charges under a 
    pseudonym?\footnote{Casebook p. 637 ff.}
    \item Does the plaintiff's privacy right outweigh judicial openness?
    \item Held: in a civil context, the victim does not have a right to use a 
    pseudonym.
\end{enumerate}

\subsubsection{The Freedom of Information Act}

\begin{enumerate}
    \item Transparency of federal documents is the default.
    \item Nine exemptions: casebook pp. 642--43.
    \item Two privacy exemptions.\footnote{Casebook p. 643.} There's a higher 
    threshold for law enforcement files.
\end{enumerate}

\paragraph{Rap Sheets: \emph{DOJ v. Reporters Committee for Freedom of the 
Press}}

FOIA does not compel disclosure of rap sheets. They have ``practical 
obscurity'' and reveal little about ``what the government is up to.''

\begin{enumerate}
    \item Much rap sheet information is public, but not in the complete form 
    that the DOJ maintains---hence the ``practical obscurity of rap sheets.''
    \item There's a high interest in maintaining practical obscurity. 
    Moreover, personal rap sheets have little to do with ``what the government 
    is up to.''
    \item Held: FOIA does not compel disclosure of rap sheets.
\end{enumerate}

\paragraph{Family Privacy vs. Government Misconduct: \emph{NARA v. Favish}}

Disclosure outweighs privacy interests only when there is reasonable evidence 
of government impropriety.

\begin{enumerate}
    \item Allan Favish sought disclosure from NARA of death-scene photographs 
    of Vincent Foster, Jr., deputy counsel to President Clinton, whose death 
    was ruled a suicide.
    \item Held: none of the photos can be released because (1) exemption 7(C) 
    is broad enough to protect a family's privacy interest in restricting 
    photos of a deceased relative, and (2) the family's privacy interest 
    outweighs the public interest.
    \item ``Only when the FOIA requester has produced \textbf{evidence 
    sufficient to warrant a belief by a reasonable person that the alleged 
    Government impropriety might have occurred} will there be a counterweight 
    on the FOIA scale for a court to balance against the cognizable privacy 
    interests in the requested documents.''
\end{enumerate}

\subsubsection{Constitutional Limitations on Public Access}

\paragraph{Arrest Records: \emph{Paul v. Davis}}

\begin{enumerate}
    \item States can publish arrest records.\footnote{Casebook p. 678.}
\end{enumerate}

\paragraph{Privacy in Criminal Records: \emph{Cline v. Rogers}}

\begin{enumerate}
    \item ``~.~.~.~there is no constitutional right to privacy in one's 
    criminal record.''\footnote{Casebook p. 681.}
\end{enumerate}

\paragraph{Police Records after Whalen: \emph{Scheetz v. The Morning Call, Inc.}}

\begin{enumerate}
    \item After \emph{Whalen}, information in police records is not 
    private.\footnote{Casebook p. 682--84.}
\end{enumerate}

\paragraph{Megan's Laws: \emph{Paul P. v. Verniero}}

\begin{enumerate}
    \item Megan's Laws require public disclosure of sex offender locations.
    \item Here, the plaintiff challenged the constitutionality of the NJ 
    Megan's Law.
    \item Held: ``Megan's Law does not restrict plaintiffs' freedom of action 
    with respect to their families.''\footnote{Casebook p. 690.}
\end{enumerate}

\subsection{Government Records of Personal Information}

\subsubsection{Fair Information Practices}

\begin{enumerate}
    \item FIPs are the rights and responsibilities associated with the 
    transfer and use of personal information. Typically, they assign rights to 
    individuals and responsibilities to organizations.\footnote{Casebook p. 
    699.}
\end{enumerate}

\subsubsection{The Privacy Act (1974)}

\begin{enumerate}
    \item Enacts FIPs into federal law.\footnote{Casebook p. 701.}
    \item Regulates how federal agencies collect and use personal information.
    \item Creates a private right of action.
\end{enumerate}

\paragraph{Hunting Records: \emph{Quinn v. Stone}}

\begin{enumerate}
    \item Hunting rosters are public records under the Privacy Act.
    \item Held: plaintiffs could show an adverse effect from revelation of 
    information on hunting records and time cards.
\end{enumerate}

\paragraph{Actual Damages: \emph{Doe v. Chao}}

\begin{enumerate}
    \item Under the Privacy Act, plaintiffs must prove some actual damages 
    in order to qualify for the minimum statutory award of \$1,000.
\end{enumerate}

\subsubsection{The Driver's Privacy Protection Act}

\begin{enumerate}
    \item State DMVs can disclose personal information only with opt-in (with 
    some exceptions).\footnote{Casebook p. 754.}
\end{enumerate}
