\section{Privacy and Law Enforcement}

\subsection{The Fourth Amendment and Emerging Technology}

\subsubsection{Introduction}

\begin{enumerate}
    \item % TODO 247-255
\end{enumerate}

\subsubsection{Wirtetapping, Bugging, and Beyond}

\paragraph{\emph{Olmstead v. United States}} % TODO 256

\begin{enumerate}
    \item 
\end{enumerate}

\paragraph{\emph{Lopez v. United States}} % TODO 263

\begin{enumerate}
    \item 
\end{enumerate}

\paragraph{\emph{Katz v. United States}} % TODO 265

\begin{enumerate}
    \item 
\end{enumerate}

\paragraph{\emph{United States v. White}} % TODO 272

\begin{enumerate}
    \item 
\end{enumerate}

\subsubsection{The Reasonable Expectation of Privacy Test and Emerging 
Technology}

\paragraph{\emph{Smith v. Maryland}} % TODO 276

\begin{enumerate}
    \item 
\end{enumerate}

\paragraph{\emph{United States v. Place}} % TODO 285

\begin{enumerate}
    \item 
\end{enumerate}

\paragraph{\emph{Illinois v. Caballes}} % TODO 286

\begin{enumerate}
    \item 
\end{enumerate}

\paragraph{\emph{California v. Greenwood}} % TODO 290

\begin{enumerate}
    \item 
\end{enumerate}

\paragraph{Plain View, Open Fields, and Curtilage} % TODO 292

\begin{enumerate}
    \item 
\end{enumerate}

\paragraph{\emph{Florida v. Riley}} % TODO 293

\begin{enumerate}
    \item 
\end{enumerate}

\paragraph{\emph{Dow Chemical v. United States}} % TODO 301


% dow chemical: no industrial curtilage
%     what's left of florida v riley?

\begin{enumerate}
    \item 
\end{enumerate}

\paragraph{\emph{Kyllo v. United States}} % TODO 306

% kyllo: core of fourth amendment is ``right of a man to retreat into his own 
% home and there be free from unreasonable government intrusion''

\begin{enumerate}
    \item 
\end{enumerate}

% TODO jardines, harris, jones

\subsection{Federal Electronic Surveillance Law}

\subsubsection{Section 605 of the Federal Communications Act}

\begin{enumerate}
    \item % TODO 313
\end{enumerate}

\subsubsection{Title III}

\begin{enumerate}
    \item % TODO 315
\end{enumerate}

\subsubsection{The Electronic Communications Privacy Act}

\begin{enumerate}
    \item % TODO 315-28
    \item % TODO schwartz ecpa chart
    \item Websites are only sometimes ECS providers. Normal retail sites, e.g. 
    L.L. Bean, are not ECS providers.
\end{enumerate}

\subsubsection{The Communications Assistance for Law Enforcement Act}

\begin{enumerate}
    \item % TODO 328
\end{enumerate}

\subsubsection{The USA PATRIOT Act}

\begin{enumerate}
    \item % TODO 331
\end{enumerate}

\subsection{Digital Searches and Seizures}

\subsection{Email---Interception vs. Storage: \emph{Steve Jackson Games v. 
United States Secret Service}}

The Wiretap Act's protection against ``interception'' does not apply to 
stored electronic communications. ``Electronic storage'' under the Wiretap Act 
does not include stored data.

\begin{enumerate}
    \item SJG operated a bulletin board that allowed users to exchange private 
    messages (``E-mail''). The Secret Service obtained a warrant to search the 
    servers for evidence of computer crimes. It seized 162 unread, private 
    messages.\footnote{Casebook pp. 345--46.}
    \item Plaintiffs sued for violation of the Wiretap Act (18 U.S.C. \S\ 
    2510--21) and the Stored Communications Act (\S\ 2701--11).
    \item Is seizure of sent but unread messages an ``intercept'' under \S\ 
    2511(1)(a)\footnote{Anyone who ``intentionally intercepts, endeavors to 
    intercept, or procures any other person to intercept or endeavor to 
    intercept, any wire, oral, or electronic communication.''}?
    \begin{enumerate}
        \item No. ```Electronic communication'' does not include electronic 
        storage of such communications.''\footnote{Casebook pp. 346--47.}
    \end{enumerate}
    \item However, the Secret Service is liable under the SCA.\footnote{\S\ 
    2701(a): ``(1) intentionally accesses without authorization a facility 
    through which an electronic communication service is provided; or (2) 
    intentionally exceeds an authorization to access that facility.''}
\end{enumerate}

\subsubsection{Kerr, ``The Problem of Perspective in Internet Law''}

\begin{enumerate}
    \item Does the Fourth Amendment protect stored emails?\footnote{Casebook 
    pp. 348--49.}
    \item \emph{Internal perspective}: the Internet is a virtual world. Email 
    is analogous to postal mail, so a warrant is required.
    \item \emph{External perspective}: the message passes through a third 
    party. No warrant is required to get email stored with a third party.
\end{enumerate}

% TODO theofel v farey-jones 349-51

\subsubsection{Privacy Expectations in Email Contents: \emph{United States v. 
Warshak}}

There is a Fourth Amendment reasonable expectation of privacy in the contents 
of emails. The SCA is unconstitutional to the extent that it lets government 
compel ISPs to turn over email contents without a warrant. (No other circuit 
has weighed in.)

\begin{enumerate}
    \item Warshak was indicted for several crimes related to his 
    pharmaceutical business. Law enforcement seized 27,000 of his emails from 
    his ISP without a warrant.
    \item Emails are private, just like letters.\footnote{Casebook p. 352--53.}
    \item \emph{Miller} (bank records) do not control because (1) these emails 
    were confidential communications, not business records and (2) the ISP was 
    an intermediary, not the intended recipient.\footnote{Casebook p. 354--55.}
    \item The SCA is unconstitutional to the extent that it lets government 
    compel ISPs to turn over email contents without a 
    warrant.\footnote{Casebook p. 355.}
    \item But the evidence should not be suppressed because the officers 
    relied in good faith on the SCA.
    \item Held: ``the government \emph{did} violate Warshak's Fourth Amendment 
    rights by compelling is Internet Service Provider (`ISP') to turn over the 
    contents of his emails. However, we agree that agents relied on the SCA in 
    good faith, and therefore hold that reversal is 
    unwarranted.''\footnote{Casebook p. 352.}
\end{enumerate}

\subsubsection{ISP Records: \emph{United States v. Hambrick}}

% TODO expand

\begin{enumerate}
    \item An undercover detective subpoenaed an ISP to get subscriber data for 
    someone posting criminal messages in a chat room. The subpoena was 
    invalid.
    \item The defendant argued that ECPA creates a Fourth Amendment REOP in 
    subscriber data. The court disagreed.\footnote{Casebook pp. 358--59.}
    \item There is no exclusionary rule here; civil damages are the only 
    remedy.
\end{enumerate}

\subsubsection{\emph{McVeigh v. Cohen}}

% TODO expand
- navy used social engineering to get info from aol
- navy ``asked'' without mcveigh ``telling''; mcveigh had a privacy 
expectation in his online activity. i.e., his online activity didn't count as 
``telling'' under DADT.
- court granted a suppression remedy [?] bc gov't solicited [?] aol to break the 
law. pp. 363--64.

\subsubsection{IP Addresses and URLs: \emph{U.S. v. Forrester}}

% TODO expand

- surveillance here: analogous to pen register; and no suppression remedy 
under the pen register act

\subsubsection{Keylogging: \emph{United States v. Scarfo}}

% TODO expand

\subsection{National Security and Foreign Intelligence}

\subsubsection{Is National Security Different? \emph{United States v. United 
States District Court} (the \emph{Keith} case)}

\begin{enumerate}
    \item % TODO 378
\end{enumerate}

\subsubsection{Foreign Intelligence Surveillance Act (FISA)}

\paragraph{\emph{Global Relief Foundation, Inc. v. O'Neil}} % TODO 388

\begin{enumerate}
    \item 
\end{enumerate}

\paragraph{\emph{United States v. Isa}} % TODO 390

\begin{enumerate}
    \item 
\end{enumerate}

\paragraph{\emph{The 9/11 Commission Report}} % TODO 392

\begin{enumerate}
    \item 
\end{enumerate}

\paragraph{\emph{In re Sealed Case}} % TODO 397

\begin{enumerate}
    \item 
\end{enumerate}

\subsubsection{Attorney General's FBI Guidelines}

\begin{enumerate}
    \item % TODO 404
\end{enumerate}

\subsubsection{NSA Surveillance Program}

\paragraph{Overview} % TODO 407-409

\begin{enumerate}
    \item
\end{enumerate}

\paragraph{11/30/11 FISC Order, Judge Bates}

\begin{enumerate}
    \item The FISC is satisfied that the government has fixed the defects in 
    its minimization procedures that the court had previously identified.
\end{enumerate}

\paragraph{8/29/13 FISC Order, Judge Eagan}

\begin{enumerate}
    \item The government's bulk collection of telephony metadata is consistent 
    with \S\ 215 of the USA PATRIOT Act and the Fourth Amendment.
\end{enumerate}

\subsubsection{\emph{Clapper v. Amnesty International}}
~\\\\
Plaintiffs lacked standing to challenge the FISA Amendments Act of 2008 (FAA).

\begin{enumerate}
    \item Several groups sued the government over the FAA, which authorizes 
    the surveillance of non-U.S. persons outside the U.S.
    \item Justice Alito:
    \begin{enumerate}
        \item Respondents do not have standing because:
        \begin{enumerate}
            \item The injury must be concrete, particularized, and actual or 
            imminent.
            \item The harm here was overly speculative.
            \item Expenditures in response to hypothetical harm (e.g., 
            traveling abroad to avoid having communications monitored) do not 
            ``manufacture standing.''\footnote{2.}
            \item The alleged injuries here are different than other standing 
            cases on which the respondents rely.\footnote{3.}
        \end{enumerate}
    \end{enumerate}
    \item Justice Breyer, dissenting:
    \begin{enumerate}
        \item The question was whether the injury was ``actual or 
        imminent.''\footnote{2.}
        \item There is a high likelihood that under \S\ 1881(a) the government 
        will intercept some of the plaintiffs' communications.\footnote{6, 9.} 
        The harm is not ``speculative.''\footnote{10.}
    \end{enumerate}
\end{enumerate}
