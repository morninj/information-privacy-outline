\section{Privacy and Law Enforcement}

\subsection{The Fourth Amendment and Emerging Technology}

\subsubsection{Introduction}

\begin{enumerate}
    \item \emph{Ex parte Jackson}: Fourth Amendment does not protect the 
    outside of letters.\footnote{Casebook p. 249.}
    \item \textbf{Special needs doctrine}: schools, government workplaces, and 
    certain highly regulated business.\footnote{Casebook p. 252.}
    \item Sobriety checks: ok, because they aim to protect road safety. Drug 
    violation checks: not ok, because aim to detect general criminal 
    wrongdoing.\footnote{Casebook p. 252--53.}
    \item \emph{Terry} stops: upon reasonable suspicion.\footnote{Casebook p. 
    254.}
\end{enumerate}

\subsubsection{Wiretapping, Bugging, and Beyond}

\paragraph{Phone Wiretapping: \emph{Olmstead v. United States}}
~\\\\
Early in the 20th Century, the Court read the Fourth Amendment narrowly to 
exclude protections for phone wiretapping.

\begin{enumerate}
    \item The government tapped the phones of suspected bootleggers. The issue 
    was whether the wiretapping violated the Fourth and Fifth Amendments. The 
    Court (Taft, J.) held that there had been no search or seizure, 
    distinguishing electronic wiretapping from postal searches.
    \item ``The United States takes no such care of telegraph or telephone 
    messages as of mailed sealed letters. The [Fourth] amendment does not 
    forbid what was done here. There was no searching. There was no seizure. 
    The evidence was secured by the used of the sense of hearing and that 
    only. There was no entry of the houses or offices of the defendants.''
    \item Justice Brandeis, dissenting:
    \begin{enumerate}
        \item ``The mail is a public service furnished by the government. The 
        telephone is a public service furnished by its authority. There is, in 
        essence, no difference between the sealed letter and the private 
        telephone message.''
    \end{enumerate}
\end{enumerate}

\paragraph{Secret Recordings of In-Person Conversations: \emph{Lopez v. United 
States}}
~\\\\
The Court adopted the \textbf{risk theory}: if you break the law, you run the 
risk that the offer ``will be accurately reproduced in court 
by~.~.~.~mechanical rendering.''

\begin{enumerate}
    \item No eavesdropping here.\footnote{Casebook p. 263.}
    \item The device was not planted through unlawful physical invasion.
\end{enumerate}

\paragraph{REOP Test: \emph{Katz v. United States}}
~\\\\
A search occurs when the government violates a person's reasonable 
expectation of privacy.

\begin{enumerate}
    \item Katz made a phone call about gambling in a telephone booth, which 
    the government listened to without entering the booth. The Court held that 
    it was a search.
    \item Justice Stewart:
    \begin{enumerate}
        \item \enquote{In the first place, the correct solution of Fourth 
        Amendment problems is not necessarily promoted by incantation of the 
        phrase \enquote{constitutionally protected area.} Secondly, the Fourth 
        Amendment cannot be translated into a general constitutional 
        \enquote{right to privacy.}}
        \item ``~.~.~.~the Fourth Amendment protects people, not places. What 
        a person knowingly exposes to the public, even in his own home or 
        office, is not a subject of Fourth Amendment protection. . . . But 
        what he seeks to preserve as private, even in an area accessible to 
        the public, may be constitutionally protected.''
    \end{enumerate}
    \item Justice Harlan: \enquote{~.~.~.~there is a twofold requirement, 
    first that a person have exhibited an actual (subjective) expectation of 
    privacy and, second, that the expectation be one that society is prepared 
    to recognize as \enquote{reasonable.}}
\end{enumerate}

\paragraph{Undercover Agent: \emph{United States v. White}}

\begin{enumerate}
    \item ``Inescapably, one contemplating illegal activities must realize the 
    risk that his companions may be reporting to the police.''
    \item Justice Harlan, dissenting: third-party bugging will undermine trust 
    and confidence. A warrant should be required.
\end{enumerate}

\subsubsection{The Reasonable Expectation of Privacy Test and Emerging 
Technology}

\paragraph{Third Party Doctrine: \emph{Smith v. Maryland}}
~\\\\
A person does not have a reasonable expectation of privacy in information that 
he voluntarily discloses to third parties.

\begin{enumerate}
    \item A phone company's installation of a pen register does not constitute 
    a search under the Fourth Amendment. Applying the Katz test, the Court 
    held that Katz had no legitimate expectation of privacy in the numbers he 
    dialed. He ``assumed the risk that the company would reveal the 
    information to the police~.~.~.~.''
    \item Justice Marshall, dissenting: ``In my view, whether privacy 
    expectations are legitimate within the meaning of Katz depends not on the 
    risks an individual can be presumed to accept when imparting information 
    to third parties, but on the risks he should be forced to assume in a free 
    and open society.''
\end{enumerate}

\paragraph{Canine Sniff: \emph{United States v. Place}}

\begin{enumerate}
    \item Canine sniffs are sui generis. ``~.~.~.~exposure of defendant's 
    luggage, which was located in a public place, to a trained 
    canine~.~.~.~did not constitute a `search' within the meaning of the 
    Fourth Amendment~.~.~.~.''\footnote{Casebook p. 285.}
\end{enumerate}

\paragraph{Dog Sniff II: \emph{Illinois v. Caballes}}
\label{par:caballes}

\begin{enumerate}
    \item A dog detected marijuana in the trunk during a traffic stop.
    \item Held: government conduct that only reveals the presence of 
    contraband compromises no legitimate privacy interest.
    \item Justice Souter, dissenting: dogs are fallible, and they are used to 
    gather information about private spaces.
    \item Justice Ginsburg, dissenting: use of a drug detection dog changes 
    the nature of a traffic stop.
\end{enumerate}

\paragraph{Are Dogs Fallible? \emph{Florida v. Harris}}

\begin{enumerate}
    \item If dogs are often wrong, does that preclude probable cause?
    \item Dogs' certification and training are adequate indications of their 
    reliability. (Unanimously held.)
\end{enumerate}

\paragraph{Dogs and Curtilage: \emph{Florida v. Jardines}}

\begin{enumerate}
    \item Majority (Justice Scalia): a dog sniff on the front porch of a home 
    is a search because it's a trespass.
    \item The front porch is a ``classic exemplar'' of \textbf{curtilage} (the 
    protected area around a house), and dog sniffs are an unlicensed physical 
    intrusion.
\end{enumerate}

\paragraph{Trash Bags: \emph{California v. Greenwood}}

\begin{enumerate}
    \item Police searched trash bags that Greenwood left out for 
    collection.\footnote{Casebook p. 290.}
    \item Held: no reasonable expectation of privacy in trash bags left on the 
    street.
\end{enumerate}

\paragraph{Plain View, Open Fields, and Curtilage}

\begin{enumerate}
    \item \textbf{Plain view doctrine}: no expectation of privacy in things 
    that can be seen from a public vantage point.\footnote{Casebook p. 293.}
    \item \textbf{Open fields doctrine}: no expectation of privacy in the open 
    fields a person owns---but curtilage is an exception.
\end{enumerate}

\paragraph{Aerial Surveillance: \emph{Florida v. Riley}}

\begin{enumerate}
    \item Officers inspected a backyard from a helicopter at 400 feet.
    \item Held: no REOP. No intimate details were revealed and there was no 
    disturbance.
\end{enumerate}

\paragraph{Industrial Curtilage: \emph{Dow Chemical v. United States}}

\begin{enumerate}
    \item There is no ``industrial curtilage.'' Taking photographs of an 
    industrial plant from navigable airspace does not violate the Fourth 
    Amendment.\footnote{Casebook p. 303.}
\end{enumerate}

\paragraph{Thermal Imagiging: \emph{Kyllo v. United States}}

\begin{enumerate}
    \item The government used thermal imaging to detect Kyllo's marijuana 
    growing operation inside his home.\footnote{Casebook p. 306.}
    \item Justice Scalia: ``Where, as here, the government uses a device that 
    is not in general public use, to explore details of a private home that 
    would previously have been unknowable without physical intrusion, the 
    surveillance is a Fourth Amendment `search,' and is presumptively 
    unreasonable without a warrant.''
    \item The core of the Fourth Amendment is the ``right of a man to retreat 
    into his own home and there be free from unreasonable government 
    intrusion.''
    \item Justice Stevens: this case did not involve ``intimate details'' of the 
    home. Rather, it involved \enquote{indirect deductions from `off-the-wall' 
    surveillance} of amorphous blobs.
\end{enumerate}

\subsection{Federal Electronic Surveillance Law}

\subsubsection{Section 605 of the Federal Communications Act}

\begin{enumerate}
    \item Federal Communications Act, 1934: prevented unauthorized 
    interception or divulgence of communications.\footnote{Casebook p. 313.}
    \item Did not apply to state prosecutions or bugging (i.e., non-wire 
    communications).
\end{enumerate}

\subsubsection{Title III}

\begin{enumerate}
    \item Enacted in 1968 in response to \emph{Katz}; amended in 1986 as the 
    Wiretap Act.
    \item Prevented all warrantless federal, state, and private wiretapping, 
    but allowed one-party consent.\footnote{Casebook p. 315.}
    \item Excluded wiretaps for national security purposes.
\end{enumerate}

\subsubsection{The Electronic Communications Privacy Act}

\begin{enumerate}
    \item See Schwartz and Solove, ``ECPA in a Nutshell.''
    \item Amended Title III (with the Wiretap Act) and added two new acts (SCA 
    and the Pen Register Act).
    \item Types of communications:\footnote{Casebook p. 316.}
    \begin{enumerate}
        \item \textbf{Wire communications}: travel through a wire or similar 
        medium. Must include a human voice.
        \item \textbf{Oral communications}: typically intercepted through 
        bugs.
        \item \textbf{Electronic communications}: all non-wire, non-oral 
        communications---e.g., email. 
    \end{enumerate}
    \item Statutory structure:
    \begin{enumerate}
        \item Wiretap Act.\footnote{Casebook p. 317--19.}
        \item Stored Communications Act.\footnote{Casebook p. 319--321.}
        \item Pen Register Act.\footnote{Casebook p. 322.}
    \end{enumerate}
    \item \textbf{Video}: if it's oral, it's covered by the Wiretap Act. If 
    it's just silent video, federal electronic surveillance law does not 
    apply.\footnote{Casebook p. 322.}
    \item Electronic surveillance orders under wiretap law have recently 
    expanded.\footnote{Casebook p. 323 ff.}
    \item State electronic surveillance law: many require consent of all 
    parties to a conversation.
    \item Websites are only sometimes ECS providers. Normal retail sites, e.g. 
    L.L. Bean, are not ECS providers.
\end{enumerate}

\subsubsection{The Communications Assistance for Law Enforcement Act}

\begin{enumerate}
    \item Telecom providers must assist legally authorized surveillance.
    \item Networks must be designed to telecoms can intercept communications 
    and provide them to law enforcement.
    \item VoIP qualifies.
\end{enumerate}

\subsection{Digital Searches and Seizures}

\subsubsection{Searching the Contents of Computers: \emph{United States v. 
Andrus}}

Third parties have apparent authority to consent to a search when an officer 
reasonably but erroneously thinks the third-party has authority to consent.

\begin{enumerate}
    \item Is the consent of the father of the suspect valid to search the 
    computer, even if the father didn't know the password? The officers used 
    forensic tools to let them circumvent the password 
    protection.\footnote{Casebook p. 336 ff.}
    \item Valid third-party consent can arise through the third party's actual 
    or apparent authority. The third party has apparent authority even when 
    the officer erroneously thinks the third party has authority to consent.
    \item Passwords on a computer are analogous to locks.
    \item Officers reasonably believed the father had authority to consent.
\end{enumerate}

\subsubsection{Email---Interception vs. Storage: \emph{Steve Jackson Games v. 
United States Secret Service}}

The Wiretap Act's protection against ``interception'' does not apply to 
stored electronic communications. ``Electronic communication'' under the 
Wiretap Act does not include stored data (so the SCA applies).

\begin{enumerate}
    \item SJG operated a bulletin board that allowed users to exchange private 
    messages (``E-mail''). The Secret Service obtained a warrant to search the 
    servers for evidence of computer crimes. It seized 162 unread, private 
    messages.\footnote{Casebook pp. 345--46.}
    \item Plaintiffs sued for violation of the Wiretap Act (18 U.S.C. \S\ 
    2510--21) and the Stored Communications Act (\S\ 2701--11).
    \item Is seizure of sent but unread messages an ``intercept'' under \S\ 
    2511(1)(a)\footnote{Anyone who ``intentionally intercepts, endeavors to 
    intercept, or procures any other person to intercept or endeavor to 
    intercept, any wire, oral, or electronic communication.''}?
    \begin{enumerate}
        \item No. ```Electronic communication'' does not include electronic 
        storage of such communications.''\footnote{Casebook pp. 346--47.}
    \end{enumerate}
    \item However, the Secret Service is liable under the SCA.\footnote{\S\ 
    2701(a): ``(1) intentionally accesses without authorization a facility 
    through which an electronic communication service is provided; or (2) 
    intentionally exceeds an authorization to access that facility.''}
\end{enumerate}

\subsubsection{Kerr, ``The Problem of Perspective in Internet Law''}

\begin{enumerate}
    \item Does the Fourth Amendment protect stored emails?\footnote{Casebook 
    pp. 348--49.}
    \item \emph{Internal perspective}: the Internet is a virtual world. Email 
    is analogous to postal mail, so a warrant is required.
    \item \emph{External perspective}: the message passes through a third 
    party. No warrant is required to get email stored with a third party.
\end{enumerate}

\subsubsection{Privacy Expectations in Email Contents: \emph{United States v. 
Warshak}}

There is a Fourth Amendment reasonable expectation of privacy in the contents 
of emails. The SCA is unconstitutional to the extent that it lets government 
compel ISPs to turn over email contents without a warrant. (No other circuit 
has weighed in.)

\begin{enumerate}
    \item Warshak was indicted for several crimes related to his 
    pharmaceutical business. Law enforcement seized 27,000 of his emails from 
    his ISP without a warrant.
    \item Emails are private, just like letters.\footnote{Casebook p. 352--53.}
    \item \emph{Miller} (bank records) do not control because (1) these emails 
    were confidential communications, not business records and (2) the ISP was 
    an intermediary, not the intended recipient.\footnote{Casebook p. 354--55.}
    \item The SCA is unconstitutional to the extent that it lets government 
    compel ISPs to turn over email contents without a 
    warrant.\footnote{Casebook p. 355.}
    \item But the evidence should not be suppressed because the officers 
    relied in good faith on the SCA.
    \item Held: ``the government \emph{did} violate Warshak's Fourth Amendment 
    rights by compelling is Internet Service Provider (`ISP') to turn over the 
    contents of his emails. However, we agree that agents relied on the SCA in 
    good faith, and therefore hold that reversal is 
    unwarranted.''\footnote{Casebook p. 352.}
\end{enumerate}

\subsubsection{REOP in ISP Records: \emph{United States v. Hambrick}}
~\\\\
Subscribers do not have a REOP in the subscription data they give to their 
ISPs. Moreover, there is no exclusionary rule in the SCA---only damages 
provisions.

\begin{enumerate}
    \item An undercover detective subpoenaed an ISP to get subscriber data for 
    someone posting criminal messages in a chat room. The subpoena was 
    invalid, but the government still obtained the data.
    \item The defendant argued that ECPA creates a Fourth Amendment REOP in 
    subscriber data. The court disagreed.\footnote{Casebook pp. 358--59.}
    \item There is no exclusionary rule in the SCA; civil damages are the only 
    remedy.
\end{enumerate}

\subsubsection{Suppression: \emph{McVeigh v. Cohen}}
~\\\\
Suppression is warranted if the government breaks the law to get information 
from a service provider.

\begin{enumerate}
    \item The Navy used social engineering to get information from AOL to 
    investigate a don't ask, don't tell issue. McVeigh didn't ``tell,'' but 
    the Navy launched a \enquote{search and `outing' mission} against him.
    \item Held: the government violated ECPA by failing to obtain a subpoena. 
    Moreover, it solicited AOL to break the law, since 18 U.S.C. \S\ 
    2703(c)(1)(B) puts the burden on the service provider to withhold 
    information from the government.\footnote{Casebook p. 363.}
\end{enumerate}

\subsubsection{No Suppression for IP Addresses and URLs: \emph{U.S. v. 
Forrester}}

\begin{enumerate}
    \item The use of a pen register is not a Fourth Amendment 
    search.\footnote{Casebook p. 366 ff.}
    \item The collection of Internet metadata here was constitutionally 
    indistinguishable from pen register collection. The Pen Register Act does 
    not provide for suppression, so there was no suppression here.
\end{enumerate}

\subsubsection{Keylogging: \emph{United States v. Scarfo}}

\begin{enumerate}
    \item Agents installed a physical keylogger.
    \item The keylogger was only activated when the modem was turned off, so 
    it did not ``intercept'' a wire communication.
\end{enumerate}

\subsection{National Security and Foreign Intelligence}

\subsubsection{Warrants for Domestic Surveillance: \emph{United States v. 
United States District Court} (the \emph{Keith} case)}

\begin{enumerate}
    \item Three categories of domestic security activity:
    \begin{enumerate}
        \item Criminal investigations.
        \item Domestic security investigations.
        \item Foreign security investigations.
    \end{enumerate}
    \item Domestic surveillance for national security risks infringing 
    ``privacy of speech,'' so a warrant is required.\footnote{Casebook p. 
    380.}
    \item Foreign surveillance for national security purposes may be 
    different.\footnote{Casebook p. 381.}
\end{enumerate}

\subsubsection{Foreign Intelligence Surveillance Act (FISA)}

\paragraph{Overview}

\begin{enumerate}
    \item FISA applies when foreign intelligence gathering is a ``significant 
    purpose'' of the investigation. (Otherwise, ECPA 
    applies.)\footnote{Casebook p. 385--86.}
    \item \textbf{National Security Letters}: the FBI can use them to get 
    information from third parties, including under the Pen Register Act and 
    SCA (but not the wiretap act). Includes a gag order provision.
\end{enumerate}

\paragraph{Emergency Exception to FISA: \emph{Global Relief Foundation, Inc. 
v. O'Neil}}

\begin{enumerate}
    \item The emergency FISA exception allows warrantless 
    searches.\footnote{Casebook p. 388.}
\end{enumerate}

\paragraph{The Wall: \emph{United States v. Isa}}

\begin{enumerate}
    \item FISA authorizes retention of evidence that is ``evidence of a 
    crime.'' The crime need not be related to foreign intelligence---as long 
    as foreign intelligence gathering was ``a significant purpose'' of the 
    investigation.
\end{enumerate}

\paragraph{Foreign Intelligence as Criminal Evidence: \emph{In re Sealed 
Case}}

\begin{enumerate}
    \item As long as a ``significant purpose'' of the investigation is 
    gathering foreign intelligence, the evidence acquired can be used in a 
    criminal case.\footnote{Casebook p. 397.}
\end{enumerate}

\subsubsection{NSA Surveillance Program}

\paragraph{11/30/11 FISC Order, Judge Bates}

\begin{enumerate}
    \item The FISC is satisfied that the government has fixed the defects in 
    its minimization procedures that the court had previously identified.
\end{enumerate}

\paragraph{8/29/13 FISC Order, Judge Eagan}

\begin{enumerate}
    \item The government's bulk collection of telephony metadata is consistent 
    with \S\ 215 of the USA PATRIOT Act and the Fourth Amendment.
\end{enumerate}

\subsubsection{\emph{Clapper v. Amnesty International}}
~\\\\
Plaintiffs lacked standing to challenge the FISA Amendments Act of 2008 (FAA).

\begin{enumerate}
    \item Several groups sued the government over the FAA, which authorizes 
    the surveillance of non-U.S. persons outside the U.S.
    \item Justice Alito:
    \begin{enumerate}
        \item Respondents do not have standing because:
        \begin{enumerate}
            \item The injury must be concrete, particularized, and actual or 
            imminent.
            \item The harm here was overly speculative.
            \item Expenditures in response to hypothetical harm (e.g., 
            traveling abroad to avoid having communications monitored) do not 
            ``manufacture standing.''\footnote{2.}
            \item The alleged injuries here are different than other standing 
            cases on which the respondents rely.\footnote{3.}
        \end{enumerate}
    \end{enumerate}
    \item Justice Breyer, dissenting:
    \begin{enumerate}
        \item The question was whether the injury was ``actual or 
        imminent.''\footnote{2.}
        \item There is a high likelihood that under \S\ 1881(a) the government 
        will intercept some of the plaintiffs' communications.\footnote{6, 9.} 
        The harm is not ``speculative.''\footnote{10.}
    \end{enumerate}
\end{enumerate}
