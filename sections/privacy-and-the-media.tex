\section{Privacy and the Media}

\subsection{Information Gathering}

\subsubsection{Intrusion upon Seclusion}

\paragraph{Restatement (Second) of Torts \S\ 652(b)}

% FIXME 

\paragraph{\emph{Nader v. General Motors Corp.}}

% FIXME 80-85
    % motion for summary judgment: would you have a case if the allegations were true?
    % two causes of action were NOT actionable:
        % entrapment: tempt him with girls
        % threatening/harassing calls [maybe court is worried that this tort will blur into nuisance?]
    % only two WERE actionable:
        % wiretapping
        % overzealous public surveillance
    % gm later settles; nader wins massive publicity

\paragraph{Private Spaces: \emph{Dietemann v. Time, Inc.}}

% FIXME 86-88
% didnt charge for services; he was practicing in private via word of mouth; cf. desnick
% would there have been a violation if the journalists had only written an article, and not taken pictures or videos? what if they had published an exact transcript of the conversation?
% 1st amendment: journalists are not privileged above others; not a license for torts/crimes

\paragraph{Public Spaces: \emph{Desnick v. American Broadcasting Co., Inc.}}
~\\\\
Professionals assume the risk that their clients will publicize their interactions.

% FIXME 88-92
% held: no privacy violation
% trespass:
    % [what interests does the tort of trespass protect?]
    % held: this case isn't really about trespass
% distinguishing from dietemann:
    % in dietemann, ``his quackery was private.''\footnote{Casebook p. 90.}
% no violation of doctor-patient privilege
% no theft, no disruption of decorum
[beside the point] % TODO 
% is there a difference between listening and recording? does the quality of recording matter? -- p schw office hrs

\paragraph{\emph{Food Lion, Inc. v. ABC}}

% FIXME 92-94

\paragraph{\emph{Shulman v. Group W Productions, Inc.}}

\begin{enumerate}
    \item Ruth Shulman was airlifted from a car accident. The ordeal was filmed 
    for a TV show, including a cameraman in the helicopter and a microphone 
    attached to the nurse's shirt which recorded the details of their 
    conversations.\footnote{Casebook pp. 94--96.}
    \item Upon broadcast, Shulman sued for unlawful intrusion and public 
    disclosure of private facts.
    \item The trial court granted summary judgment because it found that the 
    events were newsworthy.
    \item Intrusion has two elements: (1) intrusion (2) that is highly 
    offensive.
    \item The court found (1) that Shulman had a reasonable expectation of 
    privacy in the helicopter ride and in her conversations with the nurse, and 
    (2) a jury could find that the recording was highly offensive.
    \item The defendants' conduct was not privileged. ``[T]he fact that a 
    reporter may be seeking `newsworthy' material does not in itself privilege 
    the investigatory activity.''\footnote{Casebook p. 97.}
    \item Reversed.
\end{enumerate}

\subsubsection{Paparazzi}

\paragraph{\emph{Galella v. Onassis}}

\begin{enumerate}
    \item Galella was a famously annoying paparazzo who had frequent run-ins 
    with Onassis and her family.
    \item Galella sued Onassis for false arrest, malicious prosecution, and 
    other causes of action. Onassis counterclaimed for invasion of privacy, 
    among others.\footnote{Casebook p. 99.}
    \item Galella did not seriously dispute the tort claims. The court dismissed 
    his First Amendment arguments---``[c]rimes and torts committed in news 
    gathering are not protected. There is no threat to a free press in requiring 
    its agents to act within the law~.~.~.~.''\footnote{Casebook p. 100.}
\end{enumerate}

\paragraph{California Anti-Paparazzi Act: Cal. Civ. Code \S\ 1708.8}

\begin{enumerate}
    \item The Act defines two forms of invasion of privacy:\footnote{Casebook p. 
    101.}
    \begin{enumerate}
        \item Physical invasion.
        \item Constructive invasion (using tools to capture information that 
        would not have been available without trespsass).
    \end{enumerate}
    \item No punishment for the sale or dissemination of recordings in violation 
    of the Act.\footnote{Casebook p. 102.}
    \item 2005 amendment: prohibited assault committed with the intent to 
    capture information in violation of the Act.
    \begin{enumerate}
        \item One criticism: assault is an intentional tort, so this doesn't 
        apply if the paparazzi act negligently.
    \end{enumerate}
    \item 2009 amendment: prohibited sale or dissemination if the person knows 
    the information was captured in violation of the law.
    \begin{enumerate}
        \item But media buyers may not know, or they may bury their heads in the 
        sand.
    \end{enumerate}
    \item 2010 amendment: prohibited false imprisonment (e.g., when paparazzi 
    say things to make the victims think they are not free to leave).
    \begin{enumerate}
        \item What if multiple paparazzi are involved?
    \end{enumerate}
    \item First Amendment issues:\footnote{Casebook pp. 104--06.}
    \begin{enumerate}
        \item Smolla: First Amendment should prohibit liability for intrusion in 
        public places.
        \item Chemerinsky: newsgathering should be subject to intermediate 
        scrutiny. The CA Act would survive because it protects the privacy of 
        the home.
        \item Dienes: the Act ``clearly target[s] the press.'' Because it 
        imposes a disproportionate burden, it should be subject to strict 
        scrutiny.
    \end{enumerate}
\end{enumerate}

\subsubsection{Video Voyeurism}

\begin{enumerate}
    \item What protections should people have from surveillance or intrusion in 
    public places?
\end{enumerate}

\paragraph{Video Voyeurism Prevention Act: 18 U.S.C. \S\ 1801}

\begin{enumerate}
    \item Prevents intentionally capturing images of intimate areas under 
    circumstances (1) where the person believed he could disrobe in privacy or 
    (2) where intimate areas would not be visible to the 
    public.\footnote{Casebook pp. 107--108.}
\end{enumerate}

\subsection{Disclosure of Truthful Information}

\begin{enumerate}
    \item What types of disclosure should trigger civil liability?
    \item How can liability for disclosure coexist with the First Amendment?
\end{enumerate}

\subsubsection{Public Disclosure of Private Facts}

\paragraph{Restatement (Second) of Torts \S\ 652(D): Publicity Given to Private 
Life}

\begin{enumerate}
    \item Liability exists when the matter publicized is (1) highly offensive 
    and (2) not of legitimate concern to the public.
    \item Seven states don't recognize the tort.\footnote{Casebook p. 110.}
\end{enumerate}

\paragraph{Private Matters I: \emph{Gill v. Hearst Publishing Co.}}

\begin{enumerate}
    \item Harper's published a photo of the plaintiffs in an affectionate pose 
    at their public ice cream stand.
    \item The court held that the plaintiffs voluntarily \enquote{waived their right of 
    privacy so far as this particular public pose was assumed, for 
    \enquote{There can be no privacy in that which is already 
    public.}}\footnote{Casebook p. 111.}
    \item The photograph only ``extended knowledge of the particular incident to 
    a somewhat larger public than actually witnessed it at the time of 
    occurrence.''\footnote{Casebook p. 112.}
    \item Justice Carter, dissenting:
    \begin{enumerate}
        \item The photo had no news value.
        \item Consenting to public observation by a few does not mean consent to 
        observation by millions of readers.\footnote{Casebook p. 112.}
    \end{enumerate}
\end{enumerate}

\paragraph{Private Matters II: \emph{Daily Times Democrat v. Graham}}

% TODO 115-20

\paragraph{Publicity: \emph{Miller v. Motorola, Inc.}}

% TODO 120-23

\paragraph{Newsworthiness Test I: \emph{Sipple v. Chronicle Publishing Co.}}

% TODO 123-36

\paragraph{Newsworthiness Test II: \emph{}}

% TODO 136-39

\paragraph{Newsworthiness Test III: \emph{}}

% TODO 139-46

\subsubsection{First Amendment Limitations}

\paragraph{\emph{Cox Broadcasting Corp. v. Cohn}}

% TODO 148-55

\paragraph{\emph{Shulman v. Group W. Productions, Inc.}}

% TODO 155-169

\paragraph{\emph{Bartnicki v. Vopper}}

% TODO 169-81

\subsection{Dissemination of False or Misleading Information}

\subsubsection{Defamation}

% TODO 181-205

\subsubsection{False Light}

% TODO 205-210

\subsubsection{Infliction of Emotional Distress}

% TODO 210-19
