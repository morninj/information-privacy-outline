\section{Privacy and the Media}

\subsection{Information Gathering}

\subsubsection{Intrusion upon Seclusion}

\paragraph{Restatement (Second) of Torts \S\ 652(b): Intrusion upon Seclusion}

\begin{enumerate}
    \item Occurs when one (1) intrudes (2) if the intrusion would be highly 
    offensive to a reasonable person.
\end{enumerate}

\paragraph{\emph{Nader v. General Motors Corp.}}

\begin{enumerate}
    \item GM harassed and eavesdropped on Nader after he published \emph{Unsafe 
    at Any Speed}. Nader sued for intrusion upon seclusion.
    \item Two causes of action were not actionable as intrusions upon seclusion 
    (entrapping him with girls, making threatening or harassing phone calls), 
    but two were (wiretapping, overzealous public surveillance).
    \item Information gathering becomes actionable when the information sought 
    is confidential and the conduct is unreasonably intrusive.\footnote{Casebook 
    p. 81.}
    \item Held: Nader's allegations were sufficient to withstand a motion for 
    summary judgment.\footnote{Casebook p. 82.}
    \item The case later settled, and Nader won massive publicity.
    \item Justice Brietel, concurring: courts should consider allegations 
    together, since privacy invasions can occur through ``extensive or 
    exhaustive monitoring and cataloguing of acts normally disconnected and 
    anonymous.''\footnote{Casebook p. 83.}
\end{enumerate}

\paragraph{Private Spaces: \emph{Dietemann v. Time, Inc.}}

\begin{enumerate}
    \item Dietemann was practicing quack medicine in his home (cf.  
    \emph{Desnick} below). A Time reporter secretly recorded a session and later 
    published a story.
    \item The court held that Dietemann's ``den was a sphere from which he could 
    reasonably expect to exclude eavesdropping newsmen. He shouldn't expect that 
    everything said in his home will be ``transmitted by photograph or 
    recording, in our modern world, in full living color and hi-fi to the public 
    at large~.~.~.~.''\footnote{Casebook p. 87.}
    \item Time attempted a First Amendment defense, but the ``First Amendment 
    has never been construed to accord newsmen immunity from torts or crimes 
    committed during the course of newsgathering.''\footnote{Casebook p. 87.}
\end{enumerate}

\paragraph{Public Spaces: \emph{Desnick v. American Broadcasting Co., Inc.}}
~\\\\
Professionals assume the risk that their clients will publicize their interactions.

\begin{enumerate}
    \item ABC did a show about fraudulent eye surgery involving Desnick, a 
    surgeon.
    \item The court held that although ABC had engaged in undercover 
    surveillance, there was no privacy violation because there was no invasion 
    of a space that the tort of trespass seeks to protect. It distinguished 
    \emph{Dietemann}: ``Dietemann was not in business, and did not advertise his 
    services or charge for them. His quackery was private.''\footnote{Casebook 
    p. 90.}
\end{enumerate}

% FIXME 88-92
% held: no privacy violation
% trespass:
    % [what interests does the tort of trespass protect?]
    % held: this case isn't really about trespass
% distinguishing from dietemann:
    % in dietemann, ``his quackery was private.''\footnote{Casebook p. 90.}
% no violation of doctor-patient privilege
% no theft, no disruption of decorum
[beside the point] % TODO 
% is there a difference between listening and recording? does the quality of recording matter? -- p schw office hrs

\paragraph{\emph{Food Lion, Inc. v. ABC}}

% FIXME 92-94

\paragraph{\emph{Shulman v. Group W Productions, Inc.}}

\begin{enumerate}
    \item Ruth Shulman was airlifted from a car accident. The ordeal was filmed 
    for a TV show, including a cameraman in the helicopter and a microphone 
    attached to the nurse's shirt which recorded the details of their 
    conversations.\footnote{Casebook pp. 94--96.}
    \item Upon broadcast, Shulman sued for unlawful intrusion and public 
    disclosure of private facts.
    \item The trial court granted summary judgment because it found that the 
    events were newsworthy.
    \item Intrusion has two elements: (1) intrusion (2) that is highly 
    offensive.
    \item The court found (1) that Shulman had a reasonable expectation of 
    privacy in the helicopter ride and in her conversations with the nurse, and 
    (2) a jury could find that the recording was highly offensive.
    \item The defendants' conduct was not privileged. ``[T]he fact that a 
    reporter may be seeking `newsworthy' material does not in itself privilege 
    the investigatory activity.''\footnote{Casebook p. 97.}
    \item Reversed.
\end{enumerate}

\subsubsection{Paparazzi}

\paragraph{\emph{Galella v. Onassis}}

\begin{enumerate}
    \item Galella was a famously annoying paparazzo who had frequent run-ins 
    with Onassis and her family.
    \item Galella sued Onassis for false arrest, malicious prosecution, and 
    other causes of action. Onassis counterclaimed for invasion of privacy, 
    among others.\footnote{Casebook p. 99.}
    \item Galella did not seriously dispute the tort claims. The court dismissed 
    his First Amendment arguments---``[c]rimes and torts committed in news 
    gathering are not protected. There is no threat to a free press in requiring 
    its agents to act within the law~.~.~.~.''\footnote{Casebook p. 100.}
\end{enumerate}

\paragraph{California Anti-Paparazzi Act: Cal. Civ. Code \S\ 1708.8}

\begin{enumerate}
    \item The Act defines two forms of invasion of privacy:\footnote{Casebook p. 
    101.}
    \begin{enumerate}
        \item Physical invasion.
        \item Constructive invasion (using tools to capture information that 
        would not have been available without trespsass).
    \end{enumerate}
    \item No punishment for the sale or dissemination of recordings in violation 
    of the Act.\footnote{Casebook p. 102.}
    \item 2005 amendment: prohibited assault committed with the intent to 
    capture information in violation of the Act.
    \begin{enumerate}
        \item One criticism: assault is an intentional tort, so this doesn't 
        apply if the paparazzi act negligently.
    \end{enumerate}
    \item 2009 amendment: prohibited sale or dissemination if the person knows 
    the information was captured in violation of the law.
    \begin{enumerate}
        \item But media buyers may not know, or they may bury their heads in the 
        sand.
    \end{enumerate}
    \item 2010 amendment: prohibited false imprisonment (e.g., when paparazzi 
    say things to make the victims think they are not free to leave).
    \begin{enumerate}
        \item What if multiple paparazzi are involved?
    \end{enumerate}
    \item First Amendment issues:\footnote{Casebook pp. 104--06.}
    \begin{enumerate}
        \item Smolla: First Amendment should prohibit liability for intrusion in 
        public places.
        \item Chemerinsky: newsgathering should be subject to intermediate 
        scrutiny. The CA Act would survive because it protects the privacy of 
        the home.
        \item Dienes: the Act ``clearly target[s] the press.'' Because it 
        imposes a disproportionate burden, it should be subject to strict 
        scrutiny.
    \end{enumerate}
\end{enumerate}

\subsubsection{Video Voyeurism}

\begin{enumerate}
    \item What protections should people have from surveillance or intrusion in 
    public places?
\end{enumerate}

\paragraph{Video Voyeurism Prevention Act: 18 U.S.C. \S\ 1801}

\begin{enumerate}
    \item Prevents intentionally capturing images of intimate areas under 
    circumstances (1) where the person believed he could disrobe in privacy or 
    (2) where intimate areas would not be visible to the 
    public.\footnote{Casebook pp. 107--108.}
\end{enumerate}

\subsection{Disclosure of Truthful Information}

\begin{enumerate}
    \item What types of disclosure should trigger civil liability?
    \item How can liability for disclosure coexist with the First Amendment?
\end{enumerate}

\subsubsection{Public Disclosure of Private Facts}

\paragraph{Restatement (Second) of Torts \S\ 652(D): Publicity Given to Private 
Life}

\begin{enumerate}
    \item Liability exists when the matter publicized is (1) highly offensive 
    and (2) not of legitimate concern to the public.
    \item Seven states don't recognize the tort.\footnote{Casebook p. 110.}
\end{enumerate}

\paragraph{Private Matters I---No Privacy for Events Occurring in Public: 
\emph{Gill v. Hearst Publishing Co.}}

\begin{enumerate}
    \item Harper's published a photo of the plaintiffs in an affectionate pose 
    at their public ice cream stand.
    \item The court held that the plaintiffs voluntarily \enquote{waived their right of 
    privacy so far as this particular public pose was assumed, for 
    \enquote{There can be no privacy in that which is already 
    public.}}\footnote{Casebook p. 111.}
    \item The photograph only ``extended knowledge of the particular incident to 
    a somewhat larger public than actually witnessed it at the time of 
    occurrence.''\footnote{Casebook p. 112.}
    \item Justice Carter, dissenting:
    \begin{enumerate}
        \item The photo had no news value.
        \item Consenting to public observation by a few does not mean consent to 
        observation by millions of readers.\footnote{Casebook p. 112.}
    \end{enumerate}
\end{enumerate}

\paragraph{Private Matters II---Involuntary Exposure: \emph{Daily Times Democrat 
v. Graham}}
~\\\\
Involuntary public exposure does not negate privacy protections. 

\begin{enumerate}
    \item The defendant published a photo of Graham as her dress was blown up 
    when she exited a fun house.\footnote{Casebook p. 115.}
    \item The newspaper argued that the photo was ``a matter of legitimate news 
    interest to the public~.~.~.~.''\footnote{Casebook p. 115.}
    \item The court held that the photo was embarrassing and possibly obscene.
    \item ``To hold that one who is \textbf{involuntarily and instantaneously} 
    enmeshed in an embarrassing pose forfeits her right of privacy merely 
    because she happened at the moment to be part of a public scene would be 
    illogical, wrong, and unjust.''\footnote{Casebook p. 116.}
    \item Commentary:
    \begin{enumerate}
        \item Courts vary in the privacy protections they give to information 
        disclosed to small groups of people.\footnote{Casebook p. 117--18.} Lior 
        Strahilevitz argues that protections should be based on how likely the 
        information is to be disseminated beyond a particular group. For 
        instance, someone can retain an interest in his HIV positive status even 
        if 60 others (friends, family, doctors, support group members) knew 
        about it.\footnote{Casebook pp. 118--19.}
        \item Someone who gives comments to journalists can retract consent 
        before publication.\footnote{Casebook p. 119.} Media entities can 
        disseminate already public information, but further disclosure can lead 
        to liability.\footnote{Casebook p. 120.}
    \end{enumerate}
\end{enumerate}

\paragraph{Publicity---Special Relationship to the ``Public'': \emph{Miller v. 
Motorola, Inc.}}
~\\\\
What are the boundaries of ``the public''? Can it be a small group?

\begin{enumerate}
    \item A nurse at Motorola disclosed that the plaintiff, an employee, had 
    undergone mastectomy surgery.
    \item Illinois law at the time required disclosure to be widespread and 
    written.
    \item The court here held that ``the public disclosure requirement may be 
    satisfied by proof that the plaintiff has a special relationship with the 
    `public' to whom the information is disclosed.''\footnote{Casebook p. 121.}
    \item (Many courts, and possibly the Restatement, 
    disagree.\footnote{Casebook pp. 122--23.})
    \item Commentary:
    \begin{enumerate}
        \item The widespread publicity requirement singles out broadcast media 
        for restraints that don't apply to ``gossip-mongers.''\footnote{Casebook 
        p. 123.}
        \item We often care more what a small group thinks of us.
    \end{enumerate}
\end{enumerate}

\paragraph{Newsworthiness Test I: \emph{Sipple v. Chronicle Publishing Co.}}

\begin{enumerate}
    \item Sipple thwarted an assassination attempt on President Ford. News 
    stories reported that Sipple was prominent in the San Francisco gay 
    community, outing Sipple to his family.\footnote{Casebook p. 123--24.}
    \item Sipple sued for public disclosure of private facts.
    \item The court here held (1) that the facts were not private (``hundreds of 
    people in a variety of cities''\footnote{Casebook p. 125.} knew that Sipple 
    was gay) and (2) the facts were newsworthy because they were ``prompted by 
    legitimate political considerations''\footnote{Casebook p. 126.} (e.g., did 
    Ford fail to publicly thank Sipple because of bias against gays?).
\end{enumerate}

\paragraph{What is newsworthy?}

\begin{enumerate}
    \item Courts use three tests:\footnote{Casebook p. 128.}
    \begin{enumerate}
        \item Defer to editorial judgment and make no distinction between news 
        and entertainment.
        \item Look to the ``customs and conventions of the community.''
        \item Require a ``logical nexus'' between the person and the matter of 
        legitimate public interest.
    \end{enumerate}
    \item Volokh: we should eliminate the tort of public disclosure. (If he's 
    right, is there anything that isn't relevant to fitness for 
    office?)\footnote{Casebook p. 129.}
    \item \emph{Neff} (\emph{Sports Illustrated} photo): ``A factually accurate 
    public disclosure is not tortious when connected with a newsworthy event 
    even though offensive to ordinary sensibilities.''\footnote{Casebook p. 
    130.}
    \begin{enumerate}
        \item Can this be reconciled with \emph{Graham} (involuntary exposure at 
        a fun house)? Maybe, on the basis that Graham's conduct was more 
        voluntary, or on the basis that courts are more skeptical of protecting
        ``involuntary'' conduct while the plaintiff is intoxicated.
    \end{enumerate}\end{enumerate}

\paragraph{Newsworthiness Test II: \emph{}}

% TODO 136-39

\paragraph{Newsworthiness Test III: \emph{}}

% TODO 139-46

\subsubsection{First Amendment Limitations}

\paragraph{\emph{Cox Broadcasting Corp. v. Cohn}}

% TODO 148-55

\paragraph{\emph{Shulman v. Group W. Productions, Inc.}}

% TODO 155-169

\paragraph{\emph{Bartnicki v. Vopper}}

% TODO 169-81

\subsection{Dissemination of False or Misleading Information}

\subsubsection{Defamation}

% TODO 181-205

\subsubsection{False Light}

% TODO 205-210

\subsubsection{Infliction of Emotional Distress}

% TODO 210-19
