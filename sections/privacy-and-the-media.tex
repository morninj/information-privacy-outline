\section{Privacy and the Media}

\subsection{Information Gathering}

\subsubsection{Intrusion upon Seclusion}

\paragraph{Restatement (Second) of Torts \S\ 652(b): Intrusion upon Seclusion}

\begin{enumerate}
    \item Occurs when one (1) \textbf{intrudes} (2) if the intrusion would be 
    \textbf{highly offensive} to a reasonable person.
\end{enumerate}

\paragraph{Unreasonable Intrusion: \emph{Nader v. General Motors Corp.}}
~\\\\
Information gathering becomes actionable when the information sought is 
confidential and the conduct is unreasonably intrusive.\footnote{Casebook p.  
81.} Here, the unreasonable behavior was eavesdropping/wiretapping and possibly 
overzealous surveillance.

\begin{enumerate}
    \item GM harassed and eavesdropped on Nader after he published \emph{Unsafe 
    at Any Speed}. Nader sued for intrusion upon seclusion.
    \item Two causes of action were not actionable as intrusions upon seclusion 
    (entrapping him with girls, making threatening or harassing phone calls), 
    but two were (wiretapping, overzealous public surveillance).
    \item Information gathering becomes actionable when the information sought 
    is confidential and the conduct is unreasonably intrusive.\footnote{Casebook 
    p. 81.}
    \item Held: Nader's allegations were sufficient to withstand a motion for 
    summary judgment.\footnote{Casebook p. 82.}
    \item The case later settled, and Nader won massive publicity.
    \item Justice Brietel, concurring: courts should consider allegations 
    together, since privacy invasions can occur through ``extensive or 
    exhaustive monitoring and cataloguing of acts normally disconnected and 
    anonymous.''\footnote{Casebook p. 83.}
\end{enumerate}

\paragraph{Private Spaces, Eavesdropping Reports, and the First Amendment: 
\emph{Dietemann v. Time, Inc.}}

People can reasonably expect to exclude eavesdropping reporters from their 
homes. The First Amendment does not justify or excuse the tort.

\begin{enumerate}
    \item Dietemann was practicing quack medicine in his home (cf.  
    \emph{Desnick} below). A Time reporter secretly recorded a session and later 
    published a story.
    \item The court held that Dietemann's ``den was a sphere from which he could 
    reasonably expect to exclude eavesdropping newsmen.'' He shouldn't expect 
    that everything said in his home will be ``transmitted by photograph or 
    recording, in our modern world, in full living color and hi-fi to the public 
    at large~.~.~.~.''\footnote{Casebook p. 87.}
    \item Time attempted a First Amendment defense, but the ``First Amendment 
    has never been construed to accord newsmen immunity from torts or crimes 
    committed during the course of newsgathering.''\footnote{Casebook p. 87.}
\end{enumerate}

\paragraph{Public vs. Spaces: \emph{Desnick v. American Broadcasting Co., Inc.}}
~\\\\
Professionals assume the risk that their clients will publicize their 
interactions. The Court distinguished \emph{Dietemann}: ``Dietemann was not in 
business, and did not advertise his services or charge for them. His quackery 
was private.''\footnote{Casebook p. 90.}

\begin{enumerate}
    \item ABC did a show about fraudulent eye surgery involving Desnick, a 
    surgeon.
    \item The court held that although ABC had engaged in undercover 
    surveillance, there was no privacy violation because there was no invasion 
    of a space that the tort of trespass seeks to protect. It distinguished 
    \emph{Dietemann}: ``Dietemann was not in business, and did not advertise his 
    services or charge for them. His quackery was private.''\footnote{Casebook 
    p. 90.}
\end{enumerate}

\paragraph{\emph{Food Lion, Inc. v. ABC}}

% TODO 92-94

\paragraph{Newsorthiness and Offensiveness: \emph{Shulman v. Group W 
Productions, Inc.}}
~\\\\
``[T]he fact that a reporter may be seeking `newsworthy' material does not in 
itself privilege the investigatory activity.''\footnote{Casebook p. 97.} A jury 
here could find that the recording was highly offensive.

\begin{enumerate}
    \item Ruth Shulman was airlifted from a car accident. The ordeal was filmed 
    for a TV show, including a cameraman in the helicopter and a microphone 
    attached to the nurse's shirt which recorded the details of their 
    conversations.\footnote{Casebook pp. 94--96.}
    \item Upon broadcast, Shulman sued for unlawful intrusion and public 
    disclosure of private facts.
    \item The trial court granted summary judgment because it found that the 
    events were newsworthy.
    \item Intrusion has two elements: (1) intrusion (2) that is highly 
    offensive.
    \item The court found (1) that Shulman had a reasonable expectation of 
    privacy in the helicopter ride and in her conversations with the nurse, and 
    (2) a jury could find that the recording was highly offensive.
    \item The defendants' conduct was not privileged. ``[T]he fact that a 
    reporter may be seeking `newsworthy' material does not in itself privilege 
    the investigatory activity.''\footnote{Casebook p. 97.}
    \item Reversed.
\end{enumerate}

\subsubsection{Paparazzi}

\paragraph{Torts and Paparazzi: \emph{Galella v. Onassis}}
~\\\\
``[c]rimes and torts committed in news gathering are not protected. There is no 
threat to a free press in requiring its agents to act within the 
law~.~.~.~.''\footnote{Casebook p. 100.}

\begin{enumerate}
    \item Galella was a famously annoying paparazzo who had frequent run-ins 
    with Onassis and her family.
    \item Galella sued Onassis for false arrest, malicious prosecution, and 
    other causes of action. Onassis counterclaimed for invasion of privacy, 
    among others.\footnote{Casebook p. 99.}
    \item Galella did not seriously dispute the tort claims. The court dismissed 
    his First Amendment arguments---``[c]rimes and torts committed in news 
    gathering are not protected. There is no threat to a free press in requiring 
    its agents to act within the law~.~.~.~.''\footnote{Casebook p. 100.}
\end{enumerate}

\paragraph{California Anti-Paparazzi Act: Cal. Civ. Code \S\ 1708.8}

\begin{enumerate}
    \item The Act defines \textbf{two forms of invasion of 
    privacy}:\footnote{Casebook p.  101.}
    \begin{enumerate}
        \item \textbf{Physical} invasion.
        \item \textbf{Constructive} invasion (using tools to capture information 
        that would not have been available without trespsass).
    \end{enumerate}
    \item No punishment for the sale or dissemination of recordings in violation 
    of the Act.\footnote{Casebook p. 102.}
    \item 2005 amendment: prohibited assault committed with the intent to 
    capture information in violation of the Act.
    \begin{enumerate}
        \item One criticism: assault is an intentional tort, so this doesn't 
        apply if the paparazzi act negligently.
    \end{enumerate}
    \item 2009 amendment: prohibited sale or dissemination if the person knows 
    the information was captured in violation of the law.
    \begin{enumerate}
        \item But media buyers may not know, or they may bury their heads in the 
        sand.
    \end{enumerate}
    \item 2010 amendment: prohibited false imprisonment (e.g., when paparazzi 
    say things to make the victims think they are not free to leave).
    \begin{enumerate}
        \item What if multiple paparazzi are involved?
    \end{enumerate}
    \item First Amendment issues:\footnote{Casebook pp. 104--06.}
    \begin{enumerate}
        \item Smolla: First Amendment should prohibit liability for intrusion in 
        public places.
        \item Chemerinsky: newsgathering should be subject to intermediate 
        scrutiny. The CA Act would survive because it protects the privacy of 
        the home.
        \item Dienes: the Act ``clearly target[s] the press.'' Because it 
        imposes a disproportionate burden, it should be subject to strict 
        scrutiny.
    \end{enumerate}
\end{enumerate}

\subsubsection{Video Voyeurism}

\begin{enumerate}
    \item What protections should people have from surveillance or intrusion in 
    public places?
\end{enumerate}

\paragraph{Video Voyeurism Prevention Act: 18 U.S.C. \S\ 1801}

\begin{enumerate}
    \item Prevents intentionally capturing images of intimate areas under 
    circumstances (1) where the person believed he could disrobe in privacy or 
    (2) where intimate areas would not be visible to the 
    public.\footnote{Casebook pp. 107--108.}
\end{enumerate}

\subsection{Disclosure of Truthful Information}

\begin{enumerate}
    \item What types of disclosure should trigger civil liability?
    \item How can liability for disclosure coexist with the First Amendment?
\end{enumerate}

\subsubsection{Public Disclosure of Private Facts}

\paragraph{Restatement (Second) of Torts \S\ 652(D): Publicity Given to Private 
Life}

\begin{enumerate}
    \item Liability exists when the matter publicized is (1) \textbf{highly 
    offensive} and (2) \textbf{not of legitimate concern to the public}.
    \item Seven states don't recognize the tort.\footnote{Casebook p. 110.}
\end{enumerate}

\paragraph{Private Matters I---No Privacy for Events Occurring in Public: 
\emph{Gill v. Hearst Publishing Co.}}
~\\\\
``There can be no privacy in that which is already public.''\footnote{Casebook 
p. 111.} Publishing to a wider audience doesn't matter (but the dissent 
disagrees).

\begin{enumerate}
    \item Harper's published a photo of the plaintiffs in an affectionate pose 
    at their public ice cream stand.
    \item The court held that the plaintiffs voluntarily \enquote{waived their 
    right of privacy so far as this particular public pose was assumed, for 
    \enquote{There can be no privacy in that which is already 
    public.}}\footnote{Casebook p. 111.}
    \item The photograph only ``extended knowledge of the particular incident to 
    a somewhat larger public than actually witnessed it at the time of 
    occurrence.''\footnote{Casebook p. 112.}
    \item Justice Carter, dissenting:
    \begin{enumerate}
        \item The photo had no news value.
        \item Consenting to public observation by a few does not mean consent to 
        observation by millions of readers.\footnote{Casebook p. 112.}
    \end{enumerate}
\end{enumerate}

\paragraph{Private Matters II---Involuntary Exposure: \emph{Daily Times Democrat 
v. Graham}}
~\\\\
Involuntary public exposure does not negate privacy protections. 

\begin{enumerate}
    \item The defendant published a photo of Graham as her dress was blown up 
    when she exited a fun house.\footnote{Casebook p. 115.}
    \item The newspaper argued that the photo was ``a matter of legitimate news 
    interest to the public~.~.~.~.''\footnote{Casebook p. 115.}
    \item The court held that the photo was embarrassing and possibly obscene.
    \item ``To hold that one who is \textbf{involuntarily and instantaneously} 
    enmeshed in an embarrassing pose forfeits her right of privacy merely 
    because she happened at the moment to be part of a public scene would be 
    illogical, wrong, and unjust.''\footnote{Casebook p. 116.}
    \item Commentary:
    \begin{enumerate}
        \item Courts vary in the privacy protections they give to information 
        disclosed to small groups of people.\footnote{Casebook p. 117--18.} Lior 
        Strahilevitz argues that protections should be based on how likely the 
        information is to be disseminated beyond a particular group. For 
        instance, someone can retain an interest in his HIV positive status even 
        if 60 others (friends, family, doctors, support group members) knew 
        about it. Three factors affect this likelihood: (1) how interesting the 
        information is, (2) group norms, and (3) the structure of the 
        group.\footnote{Casebook pp. 118--19.}
        \item Someone who gives comments to journalists can retract consent 
        before publication.\footnote{Casebook p. 119.} Media entities can 
        disseminate already public information, but further disclosure can lead 
        to liability.\footnote{Casebook p. 120.}
    \end{enumerate}
\end{enumerate}

\paragraph{Publicity---Special Relationship to the ``Public'': \emph{Miller v.  
Motorola, Inc.}}
~\\\\
What are the boundaries of ``the public''? Can it be a small group?  The court 
here held that ``the public disclosure requirement may be satisfied by proof 
that the plaintiff has a special relationship with the `public' to whom the 
information is disclosed,'' but many courts disagree.\footnote{Casebook p. 121.}

\begin{enumerate}
    \item A nurse at Motorola disclosed that the plaintiff, an employee, had 
    undergone mastectomy surgery.
    \item Illinois law at the time required disclosure to be widespread and 
    written.
    \item The court here held that ``the public disclosure requirement may be 
    satisfied by proof that the plaintiff has a special relationship with the 
    `public' to whom the information is disclosed.''\footnote{Casebook p. 121.}
    \item (Many courts, and possibly the Restatement, 
    disagree.\footnote{Casebook pp. 122--23.})
    \item Commentary:
    \begin{enumerate}
        \item The widespread publicity requirement singles out broadcast media 
        for restraints that don't apply to ``gossip-mongers.''\footnote{Casebook 
        p. 123.}
        \item We often care more what a small group thinks of us.
    \end{enumerate}
\end{enumerate}

\paragraph{Newsworthiness Test I: \emph{Sipple v. Chronicle Publishing Co.}}
~\\\\
There is no liability for disclosing facts that are (1) not private and (2) 
newsworthy.

\begin{enumerate}
    \item Sipple thwarted an assassination attempt on President Ford. News 
    stories reported that Sipple was prominent in the San Francisco gay 
    community, outing Sipple to his family.\footnote{Casebook p. 123--24.}
    \item Sipple sued for public disclosure of private facts.
    \item The court here held (1) that the facts were not private (``hundreds of 
    people in a variety of cities''\footnote{Casebook p. 125.} knew that Sipple 
    was gay) and (2) the facts were newsworthy because they were ``prompted by 
    legitimate political considerations''\footnote{Casebook p. 126.} (e.g., did 
    Ford fail to publicly thank Sipple because of bias against gays?).
\end{enumerate}

\paragraph{What is newsworthy?}

\begin{enumerate}
    \item Courts use \textbf{three tests}:\footnote{Casebook p. 128.}
    \begin{enumerate}
        \item \textbf{Defer to editorial judgment} and make no distinction 
        between news and entertainment.
        \item Look to the \textbf{``customs and conventions of the community.''}
        \item Require a \textbf{``logical nexus''} between the person and the 
        matter of legitimate public interest.
    \end{enumerate}
    \item Volokh: we should eliminate the tort of public disclosure. (If he's 
    right, is there anything that isn't relevant to fitness for 
    office?)\footnote{Casebook p. 129.}
    \item \emph{Neff} (\emph{Sports Illustrated} photo): ``A factually accurate 
    public disclosure is not tortious when connected with a newsworthy event 
    even though offensive to ordinary sensibilities.''\footnote{Casebook p.  
    130.}
    \begin{enumerate}
        \item Can this be reconciled with \emph{Graham} (involuntary exposure at 
        a fun house)? Maybe, on the basis that Graham's conduct was more 
        voluntary, or on the basis that courts are more skeptical of protecting
        ``involuntary'' conduct while the plaintiff is intoxicated.
    \end{enumerate}
    \item
\end{enumerate}

\paragraph{Newsworthiness Test II: \emph{Shulman v. Group W Productions}}

The test for newsworthiness is ``substantial relevance'' to newsworthy subject 
matter. The court was deferential to editorial judgment.

\begin{enumerate}
    \item Were Ruth's ``appearance and words'' of legitimate public concern?
    \item Held: yes, it was newsworthy. The topic of public interest was the 
    nurse's ability to handle the crisis.
\end{enumerate}

\paragraph{Newsworthiness Test III: \emph{Bonome v. Kaysen}}

The court applied the ``logical nexus'' test to protect the publication---even 
though it left behind significant human debris. Is anything fair game for a 
memoir?

\begin{enumerate}
    \item Kaysen wrote a book that included scenes from her romantic 
    relationship with Bonome---including scenes that attributed aggressive 
    sexual force to him.
    \item Held: the issue of public interest was the line between consent and 
    non-consensual physical intimacy. There was a logical nexus between the 
    material here and the issue of public interest.
\end{enumerate}

\subsubsection{First Amendment Limitations}

\paragraph{Disseminating Public Records: \emph{Cox Broadcasting Corp. v. Cohn}}

States can't prohibit the accurate publication of a name obtained from public 
records.

\begin{enumerate}
    \item A reporter learned the name of a rape victim from an indictment 
    available for public inspection. A news report published the victim's name.  
    The victim's parents sued for invasion of privacy, citing a Georgia statute 
    making it a misdemeanor to broadcast the name or identity of a rape 
    victim.\footnote{Casebook pp.  148--49.}
    \item Public disclosure of private facts was the tort at issue.
    \item The Court has avoided the question of whether laws can prevent 
    publication of truthful but very private facts without violating the 
    constitution.\footnote{Casebook p. 150.}
    \item The Court here (Justice White) addressed a narrower question: can 
    states prohibit the accurate publication of a name obtained from public 
    records? Held, the State could not:\footnote{Casebook pp. 150--51.}
    \begin{enumerate}
        \item Journalists serve a watchdog function. Allowing journalists to 
        public the contents of public court records is important for government 
        accountability and transparency.
        \item The state must have thought it was serving the public interest by 
        putting the information in the public domain. The First and Fourteenth 
        Amendments ``command nothing less than that the States may not impose 
        sanctions on the publication of truthful information contained in 
        official court records open to public inspection.''
        \item Another holding ``would invite timidity and self-censorship.''
    \end{enumerate}
\end{enumerate}

\paragraph{Pseudonymous Litigation: \emph{Florida Star v. B.J.F.}}

\begin{enumerate}
    \item ``[I]f a newspaper lawfully obtains truthful information about a 
    matter of public significance then state officials may not constitutionally 
    punish publication of the information, absent a need to further a state 
    interest of the highest order.''\footnote{Casebook p. 155--60.}
\end{enumerate}

\paragraph{\emph{Bartnicki v. Vopper}}
~\\\\
``~.~.~.~a stranger's illegal conduct does not suffice to remove the First 
Amendment shield from speech about a \textbf{matter of public concern} [but only 
for matters of public concern---not general conversations].''

\begin{enumerate}
    \item Somebody received an anonymous tape of a conversation in which union 
    negotiators threatened violence. The tape was recorded in violation of the 
    Wiretap Act.\footnote{Casebook p. 169 ff.}
    \item Held: if the illegally obtained communication relates to a matter of 
    public concern, the First Amendment prevents application of the Wiretap Act.  
    \item ``~.~.~.~a stranger's illegal conduct does not suffice to remove the 
    First Amendment shield from speech about a matter of public concern.''
    \item Justice Rehnquist, dissenting: this rule chills, rather than promotes, 
    free speech.\footnote{Casebook p. 173 ff.}
\end{enumerate}

\subsection{Dissemination of False or Misleading Information}

\subsubsection{Defamation}

\begin{enumerate}
    \item Defamation: \textbf{false information} that \textbf{harms the 
    reputation} of the victim. Consists of libel (written) and slander (spoken).
\end{enumerate}

\paragraph{CDA \S\ 230 and Broad Immunity for Service Providers: \emph{Zeran v.  
AOL}}
~\\\\
Computer service providers are not publishers (a category which includes 
distributors). Also, CDA \S\ 230 did not create notice-based liability for 
service providers.

\begin{enumerate}
    \item CDA \S\ 230: ``No provider or user of an interactive computer service 
    shall be treated as the publisher or speaker of any information provided by 
    another information content provider.''
    \item Somebody posted defamatory information about Zeran on an AOL message 
    board.\footnote{Casebook p. 185 ff.}
    \item Is AOL a distributor? Traditionally, distributors are not liable 
    unless they have \emph{actual knowledge} of the defamatory statements.
    \item The court rejected distributor liability. It held that computer 
    service providers are not publishers (a category which includes 
    distributors). It also held that CDA \S\ 230 did not create notice-based 
    liability for service providers.
\end{enumerate}

\paragraph{\emph{Blumenthal v. Drudge}}
~\\\\
Does CDA \S\ 230 provide too much immunity from tort liability?

\begin{enumerate}
    \item Drudge wrote defamatory content about Blumenthal in his AOL 
    column.\footnote{Casebook p. 188 ff.} Blumenthal sued Drudge and AOL (here).
    \item Held: even though AOL had editorial control, CDA \S\ 230 granted it 
    immunity from tort liability.
\end{enumerate}

\paragraph{Public Officials and Actual Malice: \emph{New York Times v.  
Sullivan}}
~\\\\
To recover for defamation, public officials must prove \textbf{actual 
malice}---i.e., knowledge that the statement was \textbf{false or made with 
reckless disregard for whether it was false or not}.

\begin{enumerate}
    \item A full-page \emph{New York Times} ad criticized Sullivan's police 
    department. Although he did not show pecuniary loss, the jury awarded 
    \$500,000 in damages for libel.
    \item Justice Brennan:
    \begin{enumerate}
        \item Can courts impose liability for libel against public officials in 
        their public capacity without abridging constitutionally protected 
        speech?\footnote{Casebook p. 195.}
        \item The speech here was constitutionally protected, but does it 
        ``forfeit[] that protection by the falsity of its 
        statements~.~.~.~''?\footnote{Casebook p. 196.}
        \item Errors are inevitable in free debate. In order for that debate to 
        have the ``breathing space'' it needs to survive, public officials can 
        only recover if they prove that the defendant made the defamatory 
        statements with \textbf{actual malice}---``with knowledge that it was 
        false or with reckless disregard of whether it was false or 
        not~.~.~.~.~''\footnote{Casebook p. 196.}
    \end{enumerate}
    \item Other Justices argued that defamation should be eliminated altogether 
    for public officials, since the truth will emerge in the marketplace of 
    ideas.\footnote{Casebook p. 197.}
\end{enumerate}

\paragraph{Actual Malice and Private Citizens: \emph{Gertz v. Robert Welch, 
Inc.}}
~\\\\
Private citizens \textbf{do not} have to prove actual malice to recover for 
actual injuries. However, they have to prove actual malice to recover 
\textbf{punitive damages}, or else juries might punish unpopular views.

\begin{enumerate}
    \item Gertz was the attorney representing a family whose son died from a 
    police shooting. In a John Birch publication, Robert Welch, Inc. falsely 
    accused Gertz of framing the officer as part of a communist 
    conspiracy.\footnote{Casebook p. 198.}
    \item Gertz sued for libel and won a \$50,000 jury verdict. But the district 
    court held that the \emph{New York Times} actual malice standard should 
    apply, and found for Welch.
    \item Justice Powell:
    \begin{enumerate}
        \item Private individuals are more vulnerable to injury from defamation 
        because they have \textbf{less access than public officials to channels 
        of communication}. Thus, protections for them are 
        greater.\footnote{Casebook p. 199.}
        \item Public officials have \textbf{assumed the risk} of public life.
        \item ``~.~.~.~private individuals are not only more vulnerable to 
        injury than public officials and public figures; they are also more 
        deserving of recovery.''\footnote{Casebook p. 199.}
        \item Held: states can determine for themselves the standard of 
        liability for defamation that injures private 
        citizens.\footnote{Casebook p. 200.}
        \item However, punitive damages require a showing of actual malice.  
        Otherwise, juries might ``use their discretion to punish expressions of 
        unpopular views.''\footnote{Casebook p. 200.} Excessive jury discretion 
        might also cause media self-censorship.
        \item Held: Gertz was not a public figure, so he did not have to prove 
        actual malice.
    \end{enumerate}
    \item Justice White, dissenting:
    \begin{enumerate}
        \item The journalism industry is powerful and unlikely to be easily 
        intimidated by the occasional defamation suit.\footnote{Casebook p.  
        201.}
    \end{enumerate}
\end{enumerate}

\paragraph{Celebrity Divorces Are Not Public Controversies: \emph{Time v.  
Firestone}}

\begin{enumerate}
    \item The court opinion in the Firestones' divorce described the couple's 
    many ``extramarital escapades.''\footnote{Casebook p. 202.} \emph{Time} 
    published an article quoting the opinion, and Mary Firestone sued for libel.
    \item Held: Firestone was not a public figure, even though she was extremely 
    wealthy. ``Dissolution of a marriage through judicial proceedings is not the 
    sort of `public controversy' referred to in 
    \emph{Gertz}~.~.~.~.~''\footnote{Casebook p. 202.}
\end{enumerate}

\paragraph{``Involuntary Limited-Purpose Public Figure'': \emph{Atlanta 
Journal-Constitution v. Jewell}}

\begin{enumerate}
    \item Jewell was the security guard who discovered a bomb at the Atlanta 
    Olympics.
    \item Held: by giving interviews, he became a ``voluntary limited-purpose 
    public figure'' by injecting himself into a news story.''
\end{enumerate}

\subsubsection{False Light}

\paragraph{Overview}

\begin{enumerate}
    \item Liability if (1) \textbf{highly offensive} to a reasonable person and 
    (2) the actor acted with knowledge or reckless disregard of the 
    falsehood.\footnote{Casebook p. 205.} Different from defamation in that no 
    harm to reputation is necessary.
\end{enumerate}

\paragraph{The First Amendment and False Light: \emph{Time, Inc. v. Hill}}
~\\\\
For matters of public concern, defendants are only liable for the false light 
tort if they acted with \textbf{knowledge of falsity or in reckless disregard 
for the truth}---i.e., actual malice. Courts are split on whether the actual 
malice standard also applies to private citizens (i.e., not public figures).

\begin{enumerate}
    \item The Hill family was held hostage in their home for 19 hours. They were 
    treated well, but a play about the event depicted violence and sexual abuse. 
    \emph{Life} published a story about the play which included re-enacted 
    violent photos taken in the actual Hill home.\footnote{Casebook p. 208.}
    \item The Hills claimed that the \emph{Life} story gave the false impression 
    that the play was accurate.
    \item Held: the First Amendment precluded application of the false light 
    statute to redress false reports of matters of public concern without proof 
    that the defendant published the report with knowledge of its falsity or in
    reckless disregard for the truth.
\end{enumerate}

\subsubsection{Infliction of Emotional Distress}

\paragraph{\emph{Hustler Magazine v. Falwell}}

% TODO 211-214

\paragraph{\emph{Snyder v. Phelps}}

% TODO 214-219

\subsection{Appropriation of Name or Likeness}

\subsubsection{Introduction}

\begin{enumerate}
    \item Appropriation: privacy-based; concerned with dignity.
    \item Right of publicity: property-based; concerned with commercial 
    reward.\footnote{Casebook p. 221.}
\end{enumerate}

\subsubsection{Name or Likeness: \emph{Carson v. Here's Johnny Portable Toilets, 
Inc.}}
~\\\\
Appropriation of identity can occur without using a name or likeness.

\begin{enumerate}
    \item Carson brought two actions, based on the right to privacy and the 
    right to publicity.
    \item Right to privacy: this tort is based on embarrassment, but there is no 
    evidence here that Carson was embarrassed.\footnote{Casebook p. 233.}
    \item Right to publicity: the ``name or likeness'' standard is too low.  
    Since the phrase ``Here's Johnny'' is so closely associated with Carson, the 
    court found an appropriation of his identity.\footnote{Casebook p. 234.}
    \item Judge Kennedy, dissenting: the court's holding allows celebrities to 
    remove phrases from the public domain forever.
    \item Courts have subsequently extended the ``name or likeness'' standard 
    much further.\footnote{Casebook pp. 225--27.}
\end{enumerate}

\subsubsection{For One's Own Use or Benefit: \emph{Raymen v. United Senior 
Association, Inc.}}

\begin{enumerate}
    \item USA, Inc. used a photo of Raymen kissing his partner to promote its 
    anti-AARP advocacy.
    \item The court held that USA used the photo to ``discuss[] policy issues,'' 
    rather than to gain commercial advantage. Since its use was noncommercial, 
    Raymen could not recover for appropriation.\footnote{Casebook p. 231.}
\end{enumerate}

\subsubsection{Connection to Matters of Public Interest: \emph{Finger v. Omni 
Publications International, Ltd.}}

\begin{enumerate}
    \item The appropriation tort does not protect against the use of one's name 
    or likeness for news, art, etc.---i.e., uses that are not purely 
    commercial.\footnote{Casebook p. 233.}
    \begin{enumerate}
        \item TODO what about the fact that most news organizations are also 
        commercial entities? and that the need to earn a profit is a major 
        factor in what they decide to report on?
    \end{enumerate}
    \item In New York, the right of the media to use someone's name or likeness 
    depends on a \textbf{``real relationship''} between the use and the article.  
    It can't be an ``advertisement in disguise.''\footnote{Casebook p. 234.}
    \item Omni published a photo of the Fingers and their six kids alongside an 
    article about the effect of caffeine on in-vitro fertilization. The court 
    held (tenuously) that the subject of the article was fertility in general; 
    thus, there was a real connection between the photograph and the 
    article.\footnote{Casebook pp. 235--36.}
\end{enumerate}

\subsubsection{First Amendment Limitations: \emph{Zacchini v. Scripps-Howard 
Broadcasting Co.}}
~\\\\
The appropriation tort does not violate the First Amendment.

\begin{enumerate}
    \item A local TV station broadcast the entirety of Zacchini's human 
    cannonball act without his consent.
    \item The Ohio Supreme Court held that the First Amendment protected the 
    broadcast. The Supreme Court granted cert to decide whether the First 
    Amendment immunized the broadcaster from liability under the appropriation 
    tort.\footnote{Casebook p. 239.}
    \item The Court drew two distinctions between the appropriation tort, at 
    issue here, and the false light tort (at issue in \emph{Time, Inc. v.  
    Hill}):\footnote{Casebook p. 240.}
    \begin{enumerate}
        \item False light is concerned with reputation, while appropriation is 
        concerned with a proprietary interest.
        \item Second, false light attempts to minimize publication, while 
        appropriation decides who gets to do the publishing.
    \end{enumerate}
    \item Held: the First and Fourteenth Amendments do not immunize the 
    broadcaster from needing to pay the performer for broadcasting the entire 
    act.\footnote{Casebook p. 240.}
\end{enumerate}

\subsubsection{Imitators: \emph{Estate of Presley v. Russen}}
~\\\\
Imitators are liable under the appropriation tort if they don't add substantial 
value.

\begin{enumerate}
    \item Russen ran a show that imitated Elvis's style. Elvis's estate sued for 
    infringement of the right of publicity.
    \item Held: the imitation show ``serves primarily to commercially exploit'' 
    Elvis's likeness without adding anything of value.\footnote{Casebook p.  
    242.}
    \item However, the court did not grant a preliminary injunction because the 
    plaintiffs could not show that continued performances would cause 
    ``immediate, irreparable harm to the commercial value of the right of 
    publicity~.~.~.~.''\footnote{Casebook p. 244.}
\end{enumerate}
