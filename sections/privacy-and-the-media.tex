\section{Privacy and the Media}

\subsection{Information Gathering}

\subsubsection{Intrusion upon Seclusion}

% FIXME rest 2nd 652b

\paragraph{\emph{Nader v. General Motors Corp.}}

% FIXME 80-85
    % motion for summary judgment: would you have a case if the allegations were true?
    % two causes of action were NOT actionable:
        % entrapment: tempt him with girls
        % threatening/harassing calls [maybe court is worried that this tort will blur into nuisance?]
    % only two WERE actionable:
        % wiretapping
        % overzealous public surveillance
    % gm later settles; nader wins massive publicity

\paragraph{Private Spaces: \emph{Dietemann v. Time, Inc.}}

% FIXME 86-88
% didnt charge for services; he was practicing in private via word of mouth; cf. desnick
% would there have been a violation if the journalists had only written an article, and not taken pictures or videos? what if they had published an exact transcript of the conversation?
% 1st amendment: journalists are not privileged above others; not a license for torts/crimes

\paragraph{Public Spaces: \emph{Desnick v. American Broadcasting Co., Inc.}}
~\\\\
Professionals assume the risk that their clients will publicize their interactions.

% FIXME 88-92
% held: no privacy violation
% trespass:
    % [what interests does the tort of trespass protect?]
    % held: this case isn't really about trespass
% distinguishing from dietemann:
    % in dietemann, ``his quackery was private.''\footnote{Casebook p. 90.}
% no violation of doctor-patient privilege
% no theft, no disruption of decorum
[beside the point] % TODO 
% is there a difference between listening and recording? does the quality of recording matter? -- p schw office hrs

\paragraph{\emph{Food Lion, Inc. v. ABC}}

% FIXME 92-94

\paragraph{\emph{Shulman v. Group W Productions, Inc.}}

\begin{enumerate}
    \item Ruth Shulman was airlifted from a car accident. The ordeal was filmed 
    for a TV show, including a cameraman in the helicopter and a microphone 
    attached to the nurse's shirt which recorded the details of their 
    conversations.\footnote{Casebook pp. 94--96.}
    \item Upon broadcast, Shulman sued for unlawful intrusion and public 
    disclosure of private facts.
    \item The trial court granted summary judgment because it found that the 
    events were newsworthy.
    \item Intrusion has two elements: (1) intrusion (2) that is highly 
    offensive.
    \item The court found (1) that Shulman had a reasonable expectation of 
    privacy in the helicopter ride and in her conversations with the nurse, and 
    (2) a jury could find that the recording was highly offensive.
    \item The defendants' conduct was not privileged. ``[T]he fact that a 
    reporter may be seeking `newsworthy' material does not in itself privilege 
    the investigatory activity.''\footnote{Casebook p. 97.}
    \item Reversed.
\end{enumerate}

\subsubsection{Paparazzi}

% TODO 98-106

\subsubsection{Video Voyeurism}

% TODO 106-08

\subsection{Disclosure of Truthful Information}

\subsubsection{Public Disclosure of Private Facts}

% TODO 108-146

\subsubsection{First Amendment Limitations}

% TODO 146-181

\subsection{Dissemination of False or Misleading Information}

\subsubsection{Defamation}

% TODO 181-205

\subsubsection{False Light}

% TODO 205-210

\subsubsection{Infliction of Emotional Distress}

% TODO 210-19
