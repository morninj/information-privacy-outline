\section{Privacy of Financial and Commercial Data}

\subsection{The Financial Services Industry and Personal Data}

\subsubsection{Fair Credit Reporting Act}

\begin{enumerate}
    \item 1970: FCRA.
    \item Scope turns on the definition of ``consumer report.'' Most reports 
    dealing with consumer credit fall within this definition.\footnote{Casebook 
    p. 758.}
    \item Statutory requirements: see casebook pp. 758--65.
\end{enumerate}

\paragraph{\emph{Smith v. Bob Smith Chevrolet, Inc.}}

\begin{enumerate}
    \item Smith agreed to buy a car from Bob Smith. Part of the deal involved a 
    trade-in of his existing car, for which Bob Smith would assume the remainder 
    of unpaid loans. He also got a GM employee discount.\footnote{Casebook p. 
    765--77..}
    \item After the sale, Bob Smith learned that it had mistakenly doubled 
    Smith's discount. It accessed Smith's consumer report. Smith argued that Bob 
    Smith negligently and wilfully violated FCRA.\footnote{Casebook p. 766.}
    \item First, Bob Smith argued that it accessed the report as part of a 
    ``business transaction.'' The court held that Congress intended to allow 
    access to reports in this context for the purpose of determining the 
    customer's eligibility for a benefit. But in this case, the use was not in 
    connection with a standard business transaction. It was not for a reason 
    beneficial to the consumer. Smith, the buyer, did not initiate the 
    transaction under which Bob Smith accessed his consumer 
    report.\footnote{Casebook p. 767-68.}
    \item Second, Bob Smith argued that it accessed the report in connection 
    with ``collection of an account of the consumer.'' But a debt did not 
    actually exist here. Bob Smith \emph{alleged} a debt but did not prove its 
    existence.\footnote{Casebook p. 768--69..}
    \item Whether Bob Smith's noncompliance was wilful was a jury 
    question.\footnote{Casebook p. 769.}
\end{enumerate}

\paragraph{\emph{Sarver v. Experian Information Solutions}}

% TODO 772-79

\subsubsection{The Use and Disclosure of Financial Information}

\paragraph{The Breach of Confidentiality Tort and Financial Institutions}

% TODO 779

\paragraph{The Graham-Leach-Bliley Act}

% TODO 780-83

\paragraph{State Financial Regulation}

% TODO 783-85

\subsubsection{Identity Theft}

\paragraph{Identity Theft Statutes}

\begin{enumerate}
    \item Identity Theft Assumption and Deterrence Act: federal, 
    1998.\footnote{Casebook p. 786.}
    \item FCRA/FACTA---see below.
    \item More than 40 state laws.
    \item Solove: the credit system enables identity theft, e.g., by the 
    frequent use of SSNs as identifiers.\footnote{Casebook p. 787--88..}
\end{enumerate}

\newpage % TODO remove

\paragraph{Tort Law: \emph{Wolfe v. MBNA America Bank}}
~\\\\
Identity theft is foreseeable and preventable, so banks have to implements 
reasonable and cost-effective means to address it.

\begin{enumerate}
    \item Wolfe sued MBNA after his identity was stolen. MBNA moved to dismiss.
    \item MBNA had issued a credit card in Wolfe's name to the thief. Wolfe 
    argued that MBNA had a duty to identify ``the accuracy and authenticity'' of 
    the credit application.\footnote{Casebook p. 789.}
    \item MBNA relied on a South Carolina decision holding that banks are not 
    negligent if they issue credit cards on the basis of fraudulent 
    applications.\footnote{Casebook p. 790.}
    \item The court here held that the South Carolina decision was flawed 
    because it was foreseeable that injury would result from negligent issuance 
    of a credit card. Banks must take reasonable measures to prevent identity 
    theft.\footnote{Casebook p. 791.}
\end{enumerate}

\subsection{Commercial Entities and Personal Data}

\subsubsection{Governance by Tort}

\paragraph{Intrusion upon Seclusion and Appropriation: \emph{Dwyer v. American 
Express Co.}}
~\\\\
Selling cardholder data is not actionable as intrusion upon seclusion or 
appropriation.

\begin{enumerate}
    \item In a class action, American Express cardholders sued over the 
    company's practice of selling cardholder data.\footnote{Casebook p. 799.}
    \item Intrusion upon seclusion:
    \begin{enumerate}
        \item Elements:
        \begin{enumerate}
            \item Unauthorized intrustion.
            \item Offensive or objectionable.
            \item Private matter.
            \item Anguish and suffering.
        \end{enumerate}
        \item Held: plaintiffs failed to satisfy the first element because they 
        voluntarily disclosed the information to AMEX. Names and addresses are 
        disclosed (for lists of consumers with specific spending habits), but no 
        financial information is disclosed.\footnote{Casebook p. 800.}
        \item Appropriation:
        \begin{enumerate}
            \item Elements: appropriation of name or likeness without without 
            consent for another's benefit.
            \item Held: individual cardholder names do not create value for the 
            defendants; moreover, disclosure or sale does not deprive 
            cardholders of any value.\footnote{Casebook p. 801.}
        \end{enumerate}
    \end{enumerate}
    \item Is there harm in knowing patterns of consumption? See pp. 802--04.
\end{enumerate}

\paragraph{Private Investigators: \emph{Remsberg v. Docusearch, Inc.}}

\begin{enumerate}
    \item Youens hired Docusearch to get Boyer's personal information. 
    Docusearch lied to get some of the information. Youens later found and 
    killed Boyer.\footnote{Casebook p. 804.}
    \item Private citizens generally have no duty to protect others from the 
    criminal acts of third parties.\footnote{Casebook p. 805.} However, people 
    have a duty not to create risks of foreseeable harm. So, if a private 
    investigator's disclosure of information creates a foreseeable risk of 
    criminal misconduct, the investigator owes a duty of care.
    \item Stalking and identity theft are foreseeable.\footnote{Casebook p. 
    806.}
    \item Intrusion upon seclusion:
    \begin{enumerate}
        \item Somebody might have an action against a third party who obtained 
        that person's SSN from a credit reporting agency, but that person must 
        prove that the intrusion would have been offensive to a reasonable 
        person.
        \item What about obtaining a work address through a pretextual phone 
        call? If the address is readily available to the public, it's not 
        private and so there cannot be an action for intrusion upon seclusion.
    \end{enumerate}
    \item Appropriation:
    \begin{enumerate}
        \item Not actionable if used for a purpose other than taking advantage 
        of the person's reputation. Here, there was no taking 
        advantage.\footnote{Casebook p. 807.}
    \end{enumerate}
\end{enumerate}

\subsubsection{Governance by Contract and Promises}

\paragraph{\emph{In re Northwest Airlines Privacy Litigation}}
~\\\\
Airlines are not prohibited from disclosing passenger records to government 
agencies.

\begin{enumerate}
    \item After 9/11, Northwest began giving Passenger Name Records (``PNRs'') 
    to NASA. Plaintiffs brought multiple claims.
    \item ECPA:\footnote{Casebook p. 814--15.}
    \begin{enumerate}
        \item \S\ 2701: no. It prevent improper \emph{access} but not improper 
        disclosure.
        \item \S\ 2702: no. Northwest was not an electronic communications 
        service provider.
    \end{enumerate}
    \item Trespass: no. The PNRs were not plaintiffs' property.
    \item Intrusion upon seclusion? No. There was no intrusion because 
    plaintiffs voluntarily conveyed the information.
    \item Contract/warranty? No.
\end{enumerate}

\paragraph{FTC Enforcement: \emph{In the Matter of Google, Inc.}}

\begin{enumerate}
    \item Google's agreement with Gmail users promised that it would seek 
    consent before using the data for other purposes.
    \item Google's use of the data for Buzz without consent was a deceptive 
    practice.\footnote{Casebook p. 823 ff.}
    \item Order: see casebook pp. 824--26. Includes the establishment of a 
    ``comprehensive privacy program'' and 20 years of reporting.
\end{enumerate}

\paragraph{Google settlement in the Safari matter}

\begin{enumerate}
    \item Google promised not to track Safari users that had opted out of 
    third-party cookies, but then it circumvented the Safari cookie 
    settings.\footnote{See \url{http://www.ftc.gov/opa/2012/08/google.shtm}.}
    \item Google and the FTC reached a \$22.5 settlement (see below).
\end{enumerate}

\paragraph{District Court order accepting the Google/FTC settlement}

\begin{enumerate}
    \item Google denies any violation, but agrees to (1) pay a \$22.5 million 
    fine, delete cookies on Safari browsers, and report on compliance to the 
    FTC.
\end{enumerate}

\paragraph{VPPA I: \emph{Dirkes v. Borough of Runnemede}}

\begin{enumerate}
    \item Anyone in possession of information that was improperly released is 
    liable under VPPA---but see \emph{Daniel} below.\footnote{Casebook p. 841.}
\end{enumerate}

\paragraph{VPPA II: \emph{Daniel v. Cantell}}

\begin{enumerate}
    \item Only video tape service providers are liable under 
    VPPA.\footnote{Casebook p. 845.}
\end{enumerate}

\paragraph{Cable Communications Policy Act}

\begin{enumerate}
    \item Applies to cable operators and service providers.\footnote{Casebook p. 
    846.}
\end{enumerate}

\paragraph{Children's Online Privacy Protection Act}

\begin{enumerate}
    \item Passed in 1998.\footnote{Casebook p. 847.}
    \item Safe harbor: no COPPA liability if the provider follows guidelines 
    published by an FTC-approved group.\footnote{Casebook p. 848.}
\end{enumerate}

\paragraph{The Concept of PII}

\begin{enumerate}
    \item We lack a uniform definition.
    \item Three approaches:\footnote{Casebook p. 873.}
    \begin{enumerate}
        \item Tautological: PII identifies people.
        \item Non-public: any non-public information is personally identifying.
        \item Specific types: list data fields that count as PII.
    \end{enumerate}
\end{enumerate}

\paragraph{\emph{Pineda v. Williams-Sonoma Stores}}

\begin{enumerate}
    \item ZIP codes are PII.
\end{enumerate}

\subsection{Data Security}

\subsubsection{Data Security Breach Notification Statutes}

\begin{enumerate}
    \item California was the first after the ChoicePoint breach; now, almost all 
    states have them.
\end{enumerate}

\subsubsection{Civil Liability}

\paragraph{\emph{Pisciotta v. Old National Bancorp}}

\begin{enumerate}
    \item After a data breach, plaintiffs requested damages for the cost of 
    credit monitoring services. The court held that there had been no actual 
    compensable injury. There were only ``allegations of increased risk of 
    future identity theft~.~.~.~.''\footnote{Casebook p. 887.}
\end{enumerate}

\paragraph{FTC TRENDnet Settlement and Order}

\begin{enumerate}
    \item TRENDnet sold IP cams. It implemented faulty security, allowing 
    hackers to access cameras that users thought were private.
    \item TRENDnet was ordered to:
    \begin{enumerate}
        \item Implement a ``comprehensive security program.''
        \item Get third-party assessments and submit reports for 20 years.
        \item Notify affected customers.
        \item File a report detailing compliance.
    \end{enumerate}
\end{enumerate}
