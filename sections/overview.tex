\section{Overview}

\subsection{Privacy and the Media}

\begin{enumerate}
    \item \textbf{Torts and the First Amendment}: see PLF p. 52.
    \item \textbf{Information gathering}.
    \begin{enumerate}
        \item \textbf{Intrusion upon seclusion}.
        \begin{enumerate}
            \item Occurs when one (1) \textbf{intrudes} (2) if the intrusion 
            would be \textbf{highly offensive} to a reasonable person. 
            \emph{Nader v. GM}.
            \item People can reasonably expect to exclude eavesdropping 
            reporters from their \textbf{homes}. The First Amendment does not 
            justify or excuse the tort. \emph{Dietemann v. Time}.
            \item \textbf{Professionals} assume the risk that their clients 
            will publicize their interactions. \emph{Desnick v. ABC}. The 
            Court distinguished \emph{Dietemann}: ``Dietemann was not in 
            business, and did not advertise his services or charge for them. 
            His quackery was private.''\footnote{Casebook p. 90.}
            \item \textbf{News reporting} does not justify highly offensive 
            intrusions.  \emph{Shulman v. Group W}.
        \end{enumerate}
        \item \textbf{Paparazzi}.
        \begin{enumerate}
            \item ``[c]rimes and torts committed in news gathering are not 
            protected. There is no threat to a free press in requiring its 
            agents to act within the law~.~.~.~.''\footnote{Casebook p. 100.} 
            \emph{Galella v. Onassis}.
            \item California Anti-Paparazzi Act---see 
            p.~\pageref{par:cal-paparazzi}.
        \end{enumerate}
        \item \textbf{Video Voyeurism Prevntion Act}: prevents intentionally 
        capturing images of intimate areas under circumstances (1) where the 
        person believed he could disrobe in privacy or (2) where intimate 
        areas would not be visible to the public.\footnote{Casebook pp. 
        107--108.}
    \end{enumerate}
    \item \textbf{Disclosure of truthful information}.
    \begin{enumerate}
        \item \textbf{Key questions}: when should disclosure trigger civil 
        liability? How can liability coexist with the First Amendment?
        \item \textbf{Public disclosure of private facts}.
        \begin{enumerate}
            \item Liability exists when the matter publicized is (1) 
            \textbf{highly offensive} and (2) \textbf{not of legitimate 
            concern to the public}.\footnote{Casebook p. 110.} Seven states 
            don't recognize it.
            \item ``There can be no privacy in that which is \textbf{already 
            public}.''\footnote{casebook p. 111.} \emph{Gill v. Hearst}. 
            Publishing to a wider audience doesn't matter (but the dissent 
            disagrees).
            \item \textbf{Involuntary public exposure} does not negate privacy 
            protections. \emph{Daily Times Democrat v. Graham}.
            \item What are the boundaries of ``the public''? Can it be a small 
            group? \emph{Miller v. Motorola}. The court here held that ``the 
            public disclosure requirement may be satisfied by proof that the 
            plaintiff has a special relationship with the `public' to whom the 
            information is disclosed,'' but many courts 
            disagree.\footnote{Casebook p. 121.}
            \item There is no liability for disclosing facts that are (1) not 
            private and (2) newsworthy. \emph{Sipple v. Chronicle}.
            \item \textbf{Three tests for newsworthiness} (see 
            p.~\pageref{par:newsworthiness}):
            \begin{enumerate}
                \item \textbf{Defer to editorial judgment} and make no 
                distinction between news and entertainment. (See 
                \emph{Shulman}, p.~\ref{par:shulman-newsworthiness}, holding 
                that the test is ``substantial relevance'' to a newsworthy 
                subject.)
                \item Look to the \textbf{``customs and conventions of the 
                community.''}
                \item Require a \textbf{``logical nexus''} between the person 
                and the matter of legitimate public interest. (See 
                \emph{Bonome v. Kaysen}.)
            \end{enumerate}
        \end{enumerate}
        \item \textbf{First Amendment limitations.}
        \begin{enumerate}
            \item States can't prohibit the accurate publication of a \textbf{name 
            obtained from public records}. \emph{Cox v. Cohn}.
            \item \textbf{Pseudonymous litigation}: ``[I]f a newspaper 
            lawfully obtains truthful information about a matter of public 
            significance then state officials may not constitutionally punish 
            publication of the information, absent a need to further a state 
            interest of the highest order.''\footnote{Casebook p. 155--60.} 
            \emph{Florida Star v. B.J.F.}
            \item ``~.~.~.~\textbf{a stranger's illegal conduct} does not 
            suffice to remove the First Amendment shield from speech about a 
            \textbf{matter of public concern} [but only for matters of public 
            concern---not general conversations].'' \emph{Barnicki v. Vopper}.
        \end{enumerate}
    \end{enumerate}
    \item \textbf{Dissemination of false or misleading information}.
    \begin{enumerate}
        \item \textbf{Defamation}:
        \begin{enumerate}
            \item Defined as \textbf{false information} that \textbf{harms the 
            reputation} of the victim. Consists of libel (written) and slander 
            (spoken).
            \item Computer service providers are not publishers (a category 
            which 
            includes distributors). Also, CDA \S\ 230 did not create notice-based 
            liability for service providers. \emph{Zeran v. AOL}.
            \item Does CDA \S\ 230 provide too much immunity from tort liability?  
            \textbf{Blumenthal v. Drudge}.
            \item To recover for defamation, \textbf{public officials} must prove 
            \textbf{actual malice}---i.e., knowledge that the statement was false 
            or made with reckless disregard for whether it was false or not.
            \item \textbf{Private citizens do not have to prove actual malice} to 
            recover for actual injuries. However, they have to prove actual malice 
            to recover \textbf{punitive damages}, or else juries might punish 
            unpopular views. \emph{Gertz v. Robert Welch}.
            \item Celebrity divorces are not public controversies. \emph{Time v. 
            Firestone}.
            \item People can become \textbf{``voluntary limited-purpose public 
            figure[s]''} by injecting themselves into news stories. \emph{Atlanta 
            Journal-Constitution v. Jewell}.
        \end{enumerate}
        \item \textbf{False light}:
        \begin{enumerate}
            \item Liability if (1) \textbf{highly offensive} to a reasonable 
            person and (2) the actor acted with knowledge or reckless 
            disregard of the falsehood.\footnote{Casebook p. 205.} Different 
            from defamation in that \textbf{no harm to reputation is 
            necessary}.
            \item For matters of \textbf{public concern}, defendants are only 
            liable for the false light tort if they acted with 
            \textbf{knowledge of falsity or in reckless disregard for the 
            truth}---i.e., actual malice. \emph{Time v. Hill}. Courts are 
            split on whether the actual malice standard also applies to 
            private citizens (i.e., not public figures).
        \end{enumerate}
        \item \textbf{Infliction of Emotional Distress}:
        \begin{enumerate}
            \item Liability arises when ``\textbf{extreme and outrageous 
            conduct intentionally or recklessly} causes severe emotional 
            distress.''\footnote{Casebook p. 211.}
            \item To claim intentional infliction of emotional distress from 
            published material, public figures and officials must also show 
            \emph{New York Times} malice. \emph{Hustler v. Falwell}.
            \item There is \textbf{special protection} for public speech on 
            matters of \textbf{public concern}. \emph{Snyder v. Phelps} 
            (Westboro Baptist).
        \end{enumerate}
    \end{enumerate}
    \item \textbf{Appropriation of name or likeness}.
    \begin{enumerate}
        \item Appropriation: \textbf{privacy-based}; concerned with dignity.
        \item Right of publicity: \textbf{property-based}; concerned with 
        commercial reward.\footnote{Casebook p. 221.}
        \item Appropriation of identity can occur without using a name or 
        likeness---e.g., a \textbf{catchphrase} like ``Here's Johnny!'' 
        \emph{Carson v. Here's Johnny}. Courts have interpreted ``likeness'' 
        broadly.
        \item The tort of appropriation \textbf{does not apply to 
        noncommercial use}. \emph{Raymen v. United Senior Association}. The 
        exception applies to news (\emph{Finger v.  Omni}, below), parody, 
        satire, etc. (e.g., the Beach Boys song ``Johnny Carson'').
        \item \textbf{News media} can use a person's name or likeness without 
        incurring liability as long as there is a \textbf{``real 
        relationship''} between the person and the story. \emph{Finger v. Omni 
        Publications}. (But what about the fact that most news organizations 
        are also commercial entities?)
        \item Letting a news broadcast show an \textbf{entire act} threatens 
        the economic value of the performance. The \textbf{First Amendment} 
        doesn't allow news organizations to undermine performers' publicity 
        rights. \emph{Zacchini v. Scripps-Howard}.
        \item \textbf{Imitators} are liable under the appropriation tort if 
        they don't add \textbf{substantial value}. \emph{Estate of Presley v. 
        Russen}.
    \end{enumerate}
\end{enumerate}
