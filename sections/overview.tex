\section{Overview}

\subsection{Privacy and the Media}

\begin{enumerate}
    \item \textbf{Torts and the First Amendment}: see PLF p. 52.
    \item \textbf{Information gathering}.
    \begin{enumerate}
        \item \textbf{Intrusion upon seclusion}.
        \begin{enumerate}
            \item Occurs when one (1) \textbf{intrudes} (2) if the intrusion 
            would be \textbf{highly offensive} to a reasonable person. 
            \emph{Nader v. GM}.
            \item People can reasonably expect to exclude eavesdropping 
            reporters from their \textbf{homes}. The First Amendment does not 
            justify or excuse the tort. \emph{Dietemann v. Time}.
            \item \textbf{Professionals} assume the risk that their clients 
            will publicize their interactions. \emph{Desnick v. ABC}. The 
            Court distinguished \emph{Dietemann}: ``Dietemann was not in 
            business, and did not advertise his services or charge for them. 
            His quackery was private.''\footnote{Casebook p. 90.}
            \item \textbf{News reporting} does not justify highly offensive 
            intrusions.  \emph{Shulman v. Group W}.
        \end{enumerate}
        \item \textbf{Paparazzi}.
        \begin{enumerate}
            \item ``[c]rimes and torts committed in news gathering are not 
            protected. There is no threat to a free press in requiring its 
            agents to act within the law~.~.~.~.''\footnote{Casebook p. 100.} 
            \emph{Galella v. Onassis}.
            \item California Anti-Paparazzi Act---see 
            p.~\pageref{par:cal-paparazzi}.
        \end{enumerate}
        \item \textbf{Video Voyeurism Prevntion Act}: prevents intentionally 
        capturing images of intimate areas under circumstances (1) where the 
        person believed he could disrobe in privacy or (2) where intimate 
        areas would not be visible to the public.\footnote{Casebook pp. 
        107--108.}
    \end{enumerate}
    \item \textbf{Disclosure of truthful information}.
    \begin{enumerate}
        \item \textbf{Key questions}: when should disclosure trigger civil 
        liability? How can liability coexist with the First Amendment?
        \item \textbf{Public disclosure of private facts}.
        \begin{enumerate}
            \item Liability exists when the matter publicized is (1) 
            \textbf{highly offensive} and (2) \textbf{not of legitimate 
            concern to the public}.\footnote{Casebook p. 110.} Seven states 
            don't recognize it.
            \item ``There can be no privacy in that which is \textbf{already 
            public}.''\footnote{casebook p. 111.} \emph{Gill v. Hearst}. 
            Publishing to a wider audience doesn't matter (but the dissent 
            disagrees).
            \item \textbf{Involuntary public exposure} does not negate privacy 
            protections. \emph{Daily Times Democrat v. Graham}.
            \item What are the boundaries of ``the public''? Can it be a small 
            group? \emph{Miller v. Motorola}. The court here held that ``the 
            public disclosure requirement may be satisfied by proof that the 
            plaintiff has a special relationship with the `public' to whom the 
            information is disclosed,'' but many courts 
            disagree.\footnote{Casebook p. 121.}
            \item There is no liability for disclosing facts that are (1) not 
            private and (2) newsworthy. \emph{Sipple v. Chronicle}.
            \item \textbf{Three tests for newsworthiness} (see 
            p.~\pageref{par:newsworthiness}):
            \begin{enumerate}
                \item \textbf{Defer to editorial judgment} and make no 
                distinction between news and entertainment. (See 
                \emph{Shulman}, p.~\ref{par:shulman-newsworthiness}, holding 
                that the test is ``substantial relevance'' to a newsworthy 
                subject.)
                \item Look to the \textbf{``customs and conventions of the 
                community.''}
                \item Require a \textbf{``logical nexus''} between the person 
                and the matter of legitimate public interest. (See 
                \emph{Bonome v. Kaysen}.)
            \end{enumerate}
        \end{enumerate}
        \item \textbf{First Amendment limitations.}
        \begin{enumerate}
            \item States can't prohibit the accurate publication of a \textbf{name 
            obtained from public records}. \emph{Cox v. Cohn}.
            \item \textbf{Pseudonymous litigation}: ``[I]f a newspaper 
            lawfully obtains truthful information about a matter of public 
            significance then state officials may not constitutionally punish 
            publication of the information, absent a need to further a state 
            interest of the highest order.''\footnote{Casebook p. 155--60.} 
            \emph{Florida Star v. B.J.F.}
            \item ``~.~.~.~\textbf{a stranger's illegal conduct} does not 
            suffice to remove the First Amendment shield from speech about a 
            \textbf{matter of public concern} [but only for matters of public 
            concern---not general conversations].'' \emph{Barnicki v. Vopper}.
        \end{enumerate}
    \end{enumerate}
    \item \textbf{Dissemination of false or misleading information}.
    \begin{enumerate}
        \item \textbf{Defamation}:
        \begin{enumerate}
            \item Defined as \textbf{false information} that \textbf{harms the 
            reputation} of the victim. Consists of libel (written) and slander 
            (spoken).
            \item Computer service providers are not publishers (a category 
            which 
            includes distributors). Also, CDA \S\ 230 did not create notice-based 
            liability for service providers. \emph{Zeran v. AOL}.
            \item Does CDA \S\ 230 provide too much immunity from tort liability?  
            \textbf{Blumenthal v. Drudge}.
            \item To recover for defamation, \textbf{public officials} must prove 
            \textbf{actual malice}---i.e., knowledge that the statement was false 
            or made with reckless disregard for whether it was false or not.
            \item \textbf{Private citizens do not have to prove actual malice} to 
            recover for actual injuries. However, they have to prove actual malice 
            to recover \textbf{punitive damages}, or else juries might punish 
            unpopular views. \emph{Gertz v. Robert Welch}.
            \item Celebrity divorces are not public controversies. \emph{Time v. 
            Firestone}.
            \item People can become \textbf{``voluntary limited-purpose public 
            figure[s]''} by injecting themselves into news stories. \emph{Atlanta 
            Journal-Constitution v. Jewell}.
        \end{enumerate}
        \item \textbf{False light}:
        \begin{enumerate}
            \item Liability if (1) \textbf{highly offensive} to a reasonable 
            person and (2) the actor acted with knowledge or reckless 
            disregard of the falsehood.\footnote{Casebook p. 205.} Different 
            from defamation in that \textbf{no harm to reputation is 
            necessary}.
            \item For matters of \textbf{public concern}, defendants are only 
            liable for the false light tort if they acted with 
            \textbf{knowledge of falsity or in reckless disregard for the 
            truth}---i.e., actual malice. \emph{Time v. Hill}. Courts are 
            split on whether the actual malice standard also applies to 
            private citizens (i.e., not public figures).
        \end{enumerate}
        \item \textbf{Infliction of Emotional Distress}:
        \begin{enumerate}
            \item Liability arises when ``\textbf{extreme and outrageous 
            conduct intentionally or recklessly} causes severe emotional 
            distress.''\footnote{Casebook p. 211.}
            \item To claim intentional infliction of emotional distress from 
            published material, public figures and officials must also show 
            \emph{New York Times} malice. \emph{Hustler v. Falwell}.
            \item There is \textbf{special protection} for public speech on 
            matters of \textbf{public concern}. \emph{Snyder v. Phelps} 
            (Westboro Baptist).
        \end{enumerate}
    \end{enumerate}
    \item \textbf{Appropriation of name or likeness}.
    \begin{enumerate}
        \item Appropriation: \textbf{privacy-based}; concerned with dignity.
        \item Right of publicity: \textbf{property-based}; concerned with 
        commercial reward.\footnote{Casebook p. 221.}
        \item Appropriation of identity can occur without using a name or 
        likeness---e.g., a \textbf{catchphrase} like ``Here's Johnny!'' 
        \emph{Carson v. Here's Johnny}. Courts have interpreted ``likeness'' 
        broadly.
        \item The tort of appropriation \textbf{does not apply to 
        noncommercial use}. \emph{Raymen v. United Senior Association}. The 
        exception applies to news (\emph{Finger v.  Omni}, below), parody, 
        satire, etc. (e.g., the Beach Boys song ``Johnny Carson'').
        \item \textbf{News media} can use a person's name or likeness without 
        incurring liability as long as there is a \textbf{``real 
        relationship''} between the person and the story. \emph{Finger v. Omni 
        Publications}. (But what about the fact that most news organizations 
        are also commercial entities?)
        \item Letting a news broadcast show an \textbf{entire act} threatens 
        the economic value of the performance. The \textbf{First Amendment} 
        doesn't allow news organizations to undermine performers' publicity 
        rights. \emph{Zacchini v. Scripps-Howard}.
        \item \textbf{Imitators} are liable under the appropriation tort if 
        they don't add \textbf{substantial value}. \emph{Estate of Presley v. 
        Russen}.
    \end{enumerate}
\end{enumerate}

\newpage

\subsection{Privacy and Law Enforcement}

\begin{enumerate}
    \item \textbf{Fourth Amendment}.
    \begin{enumerate}
        \item No privacy in envelope exteriors. \emph{Ex parte Jackson}.
        \item \textbf{Special needs doctrine}: schools, government workplaces, 
        and certain highly regulated business.\footnote{Casebook p. 252.}
        \item Sobriety checks: ok, because they aim to protect road safety. 
        Drug violation checks: not ok, because aim to detect general criminal 
        wrongdoing.\footnote{Casebook p. 252--53.}
        \item \emph{Terry} stops: upon reasonable suspicion.\footnote{Casebook 
        p.  254.}
        \item \textbf{Wiretapping and bugging}.
        \begin{enumerate}
            \item Early in the 20th Century, the Court read the Fourth 
            Amendment narrowly to exclude protections for phone wiretapping. 
            \emph{Olmstead v. U.S.}
            \item \textbf{Risk theory}: if you break the law, you run the risk 
            that the offer you make in-person ``will be accurately reproduced 
            in court by~.~.~.~mechanical rendering.'' \emph{Lopez v. U.S.}
            \item A search occurs when the government violates a person's 
            reasonable expectation of privacy. \emph{Katz v. U.S.}
            \item \textbf{Undercover agents}: ``Inescapably, one contemplating 
            illegal activities must realize the risk that his companions may 
            be reporting to the police.'' \emph{U.S. v. White}.
        \end{enumerate}
        \item \textbf{Reasonable expectation of privacy}.
        \begin{enumerate}
            \item No REOP in numbers dialed. \emph{Smith v. Maryland}.
            \item Canine sniffs: sui generis, and not a search. \emph{U.S. v. 
            Place}.
            \item Canine sniffs during traffic stops are not searches. 
            Government conduct that only reveals the presence of contraband 
            compromises no legitimate privacy interest. \emph{Illinois v. 
            Caballes}. (Souter and Ginsburg dissents: see 
            p.~\pageref{par:caballes}.)
            \item Dogs' certification and training are adequate indications of 
            their reliability. \emph{Florida v. Harris}.
            \item A dog sniff on the front porch of a home is a search because 
            it's a trespass. \emph{Florida v. Jardines}.
            \item No reasonable expectation of privacy in trash bags left on 
            the street. \emph{California v. Greenwood}.
            \item \textbf{Plain view doctrine}: no expectation of privacy in 
            things that can be seen from a public vantage 
            point.\footnote{Casebook p. 293.}
            \item \textbf{Open fields doctrine}: no expectation of privacy in 
            the open fields a person owns---but curtilage is an exception.
            \item Aerial surveillance: no REOP in a backyard viewed from 400 
            feet. \emph{Florida v. Riley}.
            \item No industrial curtilage. \emph{Dow Chemical v. U.S.}.
            \item Thermal imaging: ``Where, as here, the government uses a 
            \textbf{device that is not in general public use}, to explore 
            details of a private home that would previously have been 
            unknowable without physical intrusion, the surveillance is a 
            Fourth Amendment `search,' and is presumptively unreasonable 
            without a warrant.'' \emph{Kyllo v.  U.S.}.
        \end{enumerate}
    \end{enumerate}
    \item \textbf{Federal electronic surveillance law}.
    \begin{enumerate}
        \item \textbf{Section 605} of the Federal Communications Act, 1934: 
        prevented unauthorized interception or divulgence of 
        communications.\footnote{Casebook p. 313.} Did not apply to state 
        prosecutions or bugging (i.e., non-wire communications).
        \item \textbf{Title III} (of the Ombnibus Crime Control and Safe 
        Streets Act of 1968): enacted in 1968 in response to \emph{Katz}; 
        amended in 1986 as the Wiretap Act. Prevented all warrantless federal, 
        state, and private wiretapping, but allowed one-party 
        consent.\footnote{Casebook p. 315.} Excluded wiretaps for national 
        security purposes.
        \item \textbf{Electronic Communications Privacy Act}.
        \begin{enumerate}
            \item See Schwartz and Solove, ``ECPA in a Nutshell.''
            \item Types of communications:\footnote{Casebook p. 316.}
            \begin{enumerate}
                \item \textbf{Wire communications}: travel through a wire or 
                similar medium. Must include a human voice.
                \item \textbf{Oral communications}: typically intercepted 
                through bugs.
                \item \textbf{Electronic communications}: all non-wire, 
                non-oral communications---e.g., email.  \end{enumerate}
            \item Statutory structure:
            \begin{enumerate}
                \item Wiretap Act.\footnote{Casebook p. 317--19.}
                \item Stored Communications Act.\footnote{Casebook p. 
                319--321.}
                \item Pen Register Act.\footnote{Casebook p. 322.}
            \end{enumerate}
            \item \textbf{Video}: if it's oral, it's covered by the Wiretap 
            Act. If it's just silent video, federal electronic surveillance 
            law does not apply.\footnote{Casebook p. 322.}
            \item Electronic surveillance orders under wiretap law have 
            recently expanded.\footnote{Casebook p. 323 ff.}
            \item State electronic surveillance law: many require consent of 
            all parties to a conversation.
        \end{enumerate}
        \item \textbf{Communications Assistance for Law Enforcement Act}.
        \begin{enumerate}
            \item Telecom providers must assist legally authorized 
            surveillance.
            \item Networks must be designed to telecoms can intercept 
            communications and provide them to law enforcement.
            \item VoIP qualifies.
        \end{enumerate}
    \end{enumerate}
    \item \textbf{Digital searches and seizures}.
    \begin{enumerate}
        \item \textbf{Searching the contents of computers}: third parties have 
        apparent authority to consent to a search when an officer reasonably 
        but erroneously thinks the third-party has authority to consent. 
        \emph{U.S. v. Andrus}.
        \item The Wiretap Act's protection against ``interception'' does not 
        apply to stored electronic communications. \textbf{``Electronic 
        communication'' under the Wiretap Act does not include stored data} 
        (so the SCA applies). \emph{Steve Jackson Games v. U.S.S.S.}
        \item Kerr: does the Fourth Amendment protect stored 
        emails?\footnote{Casebook pp. 348--49.}
        \begin{enumerate}
            \item \emph{Internal perspective}: the Internet is a virtual 
            world. Email is analogous to postal mail, so a warrant is 
            required.
            \item \emph{External perspective}: the message passes through a 
            third party. No warrant is required to get email stored with a 
            third party.
        \end{enumerate}
        \item There is a Fourth Amendment reasonable expectation of privacy in 
        the contents of emails. The SCA is unconstitutional to the extent that 
        it lets government compel ISPs to turn over email contents without a 
        warrant. (No other circuit has weighed in.) \emph{U.S. v. Warshak}.
        \item Subscribers do not have a REOP in the \textbf{subscription data 
        they give to their ISPs}. Moreover, there is \textbf{no exclusionary 
        rule in the SCA}---only damages provisions. \emph{U.S. v. Hambrick}.
        \item Suppression is warranted if the government breaks the law to get 
        information from a service provider. \emph{McVeigh v. Cohen}.
        \item The \textbf{collection of Internet metadata is constitutionally 
        indistinguishable from pen register collection}. The Pen Register Act 
        does not provide for suppression, so there was no suppression here. 
        \emph{U.S. v. Forrester}.
        \item \emph{U.S. v. Scarfo}: since the keylogger was only activated 
        when the modem was turned off, so it did not ``intercept'' a wire 
        communication.
    \end{enumerate}
    \item \textbf{National security and foreign intelligence}.
    \begin{enumerate}
        \item Domestic surveillance for national security risks infringing 
        ``privacy of speech,'' so a warrant is required.\footnote{Casebook p.  
        380.} (But foreign surveillance for national security purposes may be 
        different.)\footnote{Casebook p. 381.} \emph{Keith} case.
        \item \textbf{FISA}: applies when foreign intelligence gathering is a 
        ``significant purpose'' of the investigation. (Otherwise, ECPA 
        applies.)\footnote{Casebook p. 385--86.}
        \item The emergency FISA exception allows warrantless searches. 
        \emph{Global Relief Foundation v. O'Neil}.
        \item \textbf{The wall}: FISA authorizes retention of evidence that is 
        ``evidence of a crime.'' The crime need not be related to foreign 
        intelligence---as long as foreign intelligence gathering was ``a 
        significant purpose'' of the investigation. \emph{U.S. v. Isa}.
        \item As long as a ``significant purpose'' of the investigation is 
        gathering foreign intelligence, the evidence acquired can be used in a 
        criminal case. \emph{In re Sealed Case}.
    \end{enumerate}
\end{enumerate}

\newpage

\subsection{Health Privacy}

\begin{enumerate}
    \item \textbf{Confidentiality of medical information}.
    \begin{enumerate}
        \item \textbf{HIPAA}.
        \begin{enumerate}
            \item Mainly about \textbf{portability}.
            \item \textbf{Privacy rule}: casebook pp. 465--68.
            \item \textbf{Security rule}: casebook pp. 468--69.
            \item \textbf{Covered entities}: health plans, clearinghouses, 
            providers.
            \item \textbf{Marketing}: authorization is required, but not for 
            the plan's own services and products.
            \item Covered entities must make \textbf{minimum necessary use and 
            disclosures}.
            \item May disclose to a \textbf{business associate} if there are 
            assurances of safeguards.
            \item CEs must implement three kinds of \textbf{safeguards}: 
            administrative, physical, and technical.
        \end{enumerate}
        \item \textbf{HITECH Act}:
        \begin{enumerate}
            \item Facilitates \textbf{electronic health records}. Increases 
            penalties and expands security rule to business associates.
            \item New \textbf{data breach notification} requirements if 
            information has been ``compromised.'' Breach notifications are 
            necessary in all situations except those in which the CE or BA 
            shows a low probability that the information has been compromised.
        \end{enumerate}
    \end{enumerate}
    \item \textbf{Constitutional protection of medical information}.
    \begin{enumerate}
        \item There are two types of privacy interests: \textbf{informational} 
        and \textbf{decisional}. \emph{Whalen v. Roe}.
        \item \textbf{Constitutional torts}: 42 U.S.C. \S\ 1983 provides civil 
        remedies for constitutional violations. Constitutional violations 
        become tort actions, enabling plaintiffs to win damages and injunctive 
        relief.\footnote{Casebook p. 510--11.} There must be a \textbf{state 
        actor}. Plaintiffs \emph{cannot} directly sue states because of the 
        Eleventh Amendment, but they \emph{can} sue any state or local 
        government official. They can also sue local governments when their 
        policy or custom inflicts the injury.\footnote{Casebook p. 510--11.}
        \item Hospital chaplains can't have open access to patient records, 
        but they can know the patient's ``basic problem.'' \emph{Carter v. 
        BMC}.
        \item To disclose a person's HIV status, the state must show a 
        compelling government interest that outweighs the substantial privacy 
        interest. \emph{Doe v. Borough of Barrington}.
        \item The seven \emph{Westinhouse} factors weight the privacy interest 
        against competing interests. \footnote{Casebook p. 520.} Interest like 
        containing healthcare costs can outweigh individual privacy interests. 
        \emph{Doe v. SEPTA}.
    \end{enumerate}
    \item \textbf{Genetic information}.
    \begin{enumerate}
        \item At least 18 state genetic privacy statutes.\footnote{Casebook p. 
        538.}
        \item DNA can be a \textbf{``future diary.''}\footnote{Casebook p. 
        539.}
        \item Issues in DNA databases---see casebook pp. 553--59.
        \item Swabbing the cheek of an arrestee to get a DNA sample is 
        reasonable under the Fourth Amendment. \emph{Maryland v. King}---see 
        p.~\pageref{sub:maryland-v-king}.
    \end{enumerate}
\end{enumerate}

\newpage

\subsection{Privacy and Government Records}

\begin{enumerate}
    \item \textbf{Public access to government records.}
    \begin{enumerate}
        \item \textbf{Public records and court records}.
        \begin{enumerate}
            \item Court records are public in all states.
            \item In civil suits, plaintiffs sometimes cannot remain 
            anonymous. However, judges have discretion. \emph{Doe v. Shakur}.
        \end{enumerate}
        \item \textbf{FOIA}.
        \begin{enumerate}
            \item Transparency of federal documents is the default. Nine 
            exemptions; two for privacy, with a higher threshold for 
            disclosure of law enforcement documents.\footnote{Casebook p. 
            643.}
            \item FOIA does not compel disclosure of rap sheets. They have 
            ``practical obscurity'' and reveal little about \textbf{``what the 
            government is up to.''} \emph{DOJ v. Reporters Committee}.
            \item Disclosure outweighs privacy interests only when there is 
            reasonable evidence of \textbf{government impropriety}. \emph{NARA 
            v.  Favish}.
        \end{enumerate}
        \item \textbf{Constitutional limits on public access}.
        \begin{enumerate}
            \item States can publish arrest records. \emph{Paul v. Davis}.
            \item ``~.~.~.~there is no constitutional right to privacy in 
            one's criminal record.'' \emph{Cline v. Rogers}.
            \item After \emph{Whalen}, information in police records is not 
            private. \emph{Scheetz v. The Morning Call}.
            \item Megan's Laws are constitutional. ``Megan's Law does not 
            restrict plaintiffs' freedom of action with respect to their 
            families.'' \emph{Paul P. v. Verniero}.
        \end{enumerate}
    \end{enumerate}
    \item \textbf{Government records of personal information.}
    \begin{enumerate}
        \item \textbf{Fair Information Practices} are the rights and 
        responsibilities associated with the transfer and use of personal 
        information. Typically, they assign rights to individuals and 
        responsibilities to organizations.\footnote{Casebook p.  699.}
        \item \textbf{Privacy Act} (1974).
        \begin{enumerate}
            \item Enacts FIPs into federal law.\footnote{Casebook p. 701.} 
            Regulates how federal agencies collect and use personal 
            information. Creates a private right of action.
            \item Hunting rosters are public records under the Privacy Act, 
            and plaintiffs may be able to show that disclosure would have 
            adverse effects. \emph{Quinn v. Stone}.
            \item Under the Privacy Act, plaintiffs must prove some 
            \textbf{actual damages} in order to qualify for the minimum 
            statutory award of \$1,000. \emph{Doe v. Chao}.
        \end{enumerate}
        \item \textbf{Driver's Privacy Protection Act}: state DMVs can 
        disclose personal information only with opt-in (with some 
        exceptions).\footnote{Casebook p. 754.}
    \end{enumerate}
\end{enumerate}

\newpage

\subsection{Privacy of Financial and Commercial Data}

\begin{enumerate}
    \item % TODO 
\end{enumerate}

\newpage

\subsection{International Privacy Law}

\begin{enumerate}
    \item % TODO
\end{enumerate}
