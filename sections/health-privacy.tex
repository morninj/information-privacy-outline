\section{Health Privacy}

\subsection{Confidentiality of Medical Information}

\subsubsection{HIPAA}

\begin{enumerate}
    \item HIPAA: 1996. HHS regulations implementing HIPAA: November 1999.  
    Primarily about \textbf{portability}.\footnote{Casebook p. 463 ff.}
    \item \textbf{Privacy rule} (2000): final version of HIPAA 
    regulations.\footnote{Casebook p. 463.} Became effective in 2003. Covers the 
    use of protected health information---see casebook pp. 465--68.
    \item \textbf{Security rule}: published 2003, effective 2005. Covers the 
    security of electronic personal health information. See casebook pp.  
    468--69.
    \item Criminal enforcement---see casebook pp. 471--73.
    \item Law enforcement access and the third party doctrine--see casebook pp.  
    473--75. The New York Court of Appeals denied a law enforcement subpoena for 
    health records, despite the fact that HIPAA (\S\ 164.512(f)) allowed 
    it.\footnote{Casebook p. 475.}
    \item \textbf{Covered entities}: health plans, clearinghouses, providers.
    \item \textbf{Marketing}: authorization is required, but not for the plan's 
    own services and products.
    \item Covered entities must make \textbf{minimum necessary use and 
    disclosures}.
    \item May disclose to a \textbf{business associate} if there are assurances 
    of safeguards.
    \item CEs must implement three kinds of \textbf{safeguards}: administrative, 
    physical, and technical.
\end{enumerate}

\subsubsection{HITECH Act}

\begin{enumerate}
    \item Facilitates electronic health records. Increases penalties and expands 
    security rule to business associates.
    \item New data breach notification requirements if information has been 
    ``compromised.'' Breach notifications are necessary in all situations except 
    those in which the CE or BA shows a low probability that the information has 
    been compromised.
\end{enumerate}

\subsection{Constitutional Protection of Medical Information}

\subsubsection{Two Privacy Interests: \emph{Whalen v. Roe}}

There are two types of privacy interests: \textbf{informational} and 
\textbf{decisional}.

\begin{enumerate}
    \item New York kept records of Schedule II drug prescriptions.
    \item Plaintiffs argued that the records would make people decline treatment 
    out of fear that the records would be misused.
    \item \textbf{Two kinds of privacy interests}:\footnote{Casebook p. 505.}
    \begin{enumerate}
        \item \textbf{Informational}: avoiding disclosure of personal matters.
        \item \textbf{Decisional}: independence in making important decisions.
    \end{enumerate}
    \item The NY statute does not threaten either enough to establish a 
    constitutional violation.
\end{enumerate}

\subsubsection{42 U.S.C. \S\ 1983 and ``Constitutional Torts''}

\begin{enumerate}
    \item Provides civil remedies for constitutional violations. Constitutional 
    violations become tort actions, enabling plaintiffs to win damages and 
    injunctive relief.\footnote{Casebook p. 510--11.}
    \item There must be a \textbf{state actor}. Plaintiffs \emph{cannot} 
    directly sue states because of the Eleventh Amendment, but they \emph{can} 
    sue any state or local government official. They can also sue local 
    governments when their policy or custom inflicts the injury.
\end{enumerate}

\subsubsection{Limited Access to Patient Records: \emph{Carter v. Broadlawns 
Medical Center}}

\begin{enumerate}
    \item BMC had a policy of allowing its chaplain open access to medical 
    records.
    \item Plaintiffs alleged the policy violated patients' confidentiality.
    \item Held: the chaplain could not have open access, but he could know about 
    the patient's ``basic problem'' (e.g., a suicide attempt).
\end{enumerate}

\subsubsection{Government Disclosure of HIV Status: \emph{Doe v. Borough of 
Barrington}}

\begin{enumerate}
    \item Police officers revealed to the defendant's neighbors the fact that he 
    was HIV positive.\footnote{Casebook p. 513.}
    \item Held: the Constitution prevents government disclosure of HIV 
    status.\footnote{Casebook p. 514.} To disclose a person's HIV status, the 
    state must show a compelling government interest that outweighs the 
    substantial privacy interest.\footnote{Casebook p. 515.}
\end{enumerate}

\subsubsection{\emph{Doe v. Southeastern Pennsylvania Transportation Authority}}

The seven \emph{Westinhouse} factors weight the privacy interest against 
competing interests.

\begin{enumerate}
    \item As part of a health insurance plan review involving auditing of 
    prescription drug records, the defendant's employer learned of the 
    defendant's HIV-positive status. He alleges he was treated differently at 
    work after the disclosure.\footnote{Casebook pp. 516--18.}
    \item Each disclosure to a new person was a separate disclosure. However, 
    disclosures to the company doctors were not actionable because they already 
    knew of the defendant's status.\footnote{Casebook p.  519.}
    \item \emph{Westinghouse}: seven factors to weight the privacy interest 
    against competing interests.
    \item Held: the intrusion here was minimal. On balance, the employer's 
    interests are more substantial--e.g., containing healthcare 
    costs.\footnote{Casebook p.  521.}
\end{enumerate}

\subsection{Genetic Information}

\subsubsection{Overview}

\begin{enumerate}
    \item Background on DNA---see casebook pp. 526--27.
    \item At least 18 state genetic privacy statutes.\footnote{Casebook p. 538.}
    \item DNA can be a \textbf{``future diary.''}\footnote{Casebook p. 539.}
    \item Issues in DNA databases---see casebook pp. 553--59.
\end{enumerate}

\subsubsection{Taking DNA Samples from Arrestees: \emph{Maryland v. King}}
\label{sub:maryland-v-king}

Swabbing the cheek of an arrestee to get a DNA sample is reasonable under the 
Fourth Amendment.

\begin{enumerate}
    \item Officers took a cheek swab after arresting King on assault charges.  
    His DNA matched samples from an unsolved rape eight years earlier. He was 
    convicted of the rape.\footnote{See slip opinion.}
    \item Is it reasonable under the Fourth Amendment to take a cheek swabbing 
    for DNA samples of arrestees?
    \item Justice Kennedy:
    \begin{enumerate}
        \item DNA testing is highly useful for law enforcement.
        \item The intrusion of a cheek swab is negligible. Legitimate government 
        interests outweigh this minimal intrusion.
        \item The government has several interests in using DNA samples:
        \begin{enumerate}
            \item Identifying who is being arrested.
            \item Other reasons---see syllabus p. 3.
        \end{enumerate}
        \item The defendant's privacy interests do not outweigh the government's 
        interests. The intrusion is minimal, the sampling does not reveal 
        genetic traits, and it is unlikely to reveal medical information.
    \end{enumerate}
    \item Justice Scalia:
    \begin{enumerate}
        \item In this case, identification of the suspect cannot possibly be the 
        purpose of taking the DNA sample. (No difference between DNA sampling 
        and fingerprinting.)
    \end{enumerate}
\end{enumerate}
