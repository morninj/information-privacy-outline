\section{Health Privacy}

\subsection{Confidentiality of Medical Information}

\begin{enumerate}
    \item HIPAA: 1996. HHS regulations implementing HIPAA: November 1999.
    \item \textbf{Privacy rule} (2000): final version of HIPAA 
    regulations.\footnote{Casebook p. 463.} Became effective in 2003. Covers the 
    use of protected health information---see casebook pp. 465--68.
    \item \textbf{Security rule}: published 2003, effective 2005. Covers the 
    security of electronic personal health information. See casebook pp. 
    468--69.
    \item Criminal enforcement---see casebook pp. 471--73.
    \item Law enforcement access and the third party doctrine--see casebook pp. 
    473--75. The New York Court of Appeals denied a law enforcement subpoena for 
    health records, despite the fact that HIPAA (\S\ 164.512(f)) allowed 
    it.\footnote{Casebook p. 475.}
\end{enumerate}

\subsection{Constitutional Protection of Medical Information}

\subsubsection{\emph{Whalen v. Roe}}

% TODO 503-10

\subsubsection{42 U.S.C. \S\ 1983 and ``Constitutional Torts''}

% TODO 510-512

\subsubsection{\emph{Carter v. Broadlawns Medical Center}}

% TODO 512

\subsubsection{\emph{Doe v. Borough of Barrington}}

% TODO expand

\begin{enumerate}
    \item Police officers revealed to the defendant's neighbors the fact that he 
    was HIV positive.\footnote{Casebook p. 513.}
    \item Held: the Constitution protects against government disclosure of HIV 
    status.\footnote{Casebook p. 514.} The state must show a compelling 
    government interest to disclose HIV status.\footnote{Casebook p. 515.}
\end{enumerate}

\subsubsection{\emph{Doe v. Southeastern Pennsylvania Transportation Authority}}

% TODO expand

\begin{enumerate}
    \item As part of a health insurance plan review involving auditing of 
    prescription drug records, the defendant's employer learned of the 
    defendant's HIV-positive status. He alleges he was treated differently at 
    work after the disclosure.\footnote{Casebook pp. 516--18.}
    \item Each disclosure to a new person was a separate 
    disclosure. However, disclosures to the company doctors were not actionable 
    because they already knew of the defendant's status.\footnote{Casebook p. 
    519.}
    \item \emph{Westinghouse}: seven factors to determine whether disclosure is 
    actionable.
    \item Held: the intrusion here was minimal. On balance, the employer's 
    interests are more substantial.\footnote{Casebook p. 521.}
\end{enumerate}

\subsection{Genetic Information}

\begin{enumerate}
    \item Background on DNA---see casebook pp. 526--27.
\end{enumerate}

\subsubsection{State Genetic Privacy Statutes}

% TODO 538-39

\subsubsection{DNA Databases}

% TODO 553-59

\subsubsection{Taking DNA Samples from Arrestees: \emph{Maryland v. King}}

Swabbing the cheek of an arrestee to get a DNA sample is reasonable under the 
Fourth Amendment.

\begin{enumerate}
    \item Officers took a cheek swab after arresting King on assault charges. 
    His DNA matched samples from an unsolved rape eight years earlier. He was 
    convicted of the rape.\footnote{See slip opinion.}
    \item Is it reasonable under the Fourth Amendment to take a cheek swabbing 
    for DNA samples of arrestees?
    \item Justice Kennedy:
    \begin{enumerate}
        \item DNA testing is highly useful for law enforcement.
        \item The intrusion of a cheek swab is negligible. Legitimate government 
        interests outweigh this minimal intrusion.
        \item The government has several interests in using DNA samples:
        \begin{enumerate}
            \item Identifying who is being arrested.
            \item Other reasons---see syllabus p. 3.
        \end{enumerate}
        \item The defendant's privacy interests do not outweigh the government's 
        interests. The intrusion is minimal, the sampling does not reveal 
        genetic traits, and it is unlikely to reveal medical information.
    \end{enumerate}
    \item Justice Scalia:
    \begin{enumerate}
        \item In this case, identification of the suspect cannot possibly be the 
        purpose of taking the DNA sample. (No difference between DNA sampling 
        and fingerprinting.)
    \end{enumerate}
\end{enumerate}
