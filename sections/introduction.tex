\section{Introduction to Information Privacy Law}

\subsection{Information Privacy, Technology, and the Law}

\subsubsection{Involuntary Public Figures and Public Interest: \emph{Sidis v. 
F-R Publishing Corp.}}

\begin{enumerate}
    \item Sidis, a former child prodigy, sued F-R over a \emph{New Yorker} 
    story by Thurber about his current life. The court held that the lives of 
    public figures are matters of public concern, so lower privacy protections 
    apply.\footnote{Casebook pp. 5--6.}
\end{enumerate}

\subsection{Information Privacy Law: Origins and Types}

\subsubsection{Common Law}

\paragraph{The Right to Be Let Alone: Warren and Brandeis, \emph{The Right to 
Privacy}}

\begin{enumerate}
    \item In response to advances in media (e.g., gossip) and technology 
    (e.g., the Kodak Brownie), which can cause ``mental pain and distress, far 
    greater than could be inflicted by mere bodily injury.''\footnote{Casebook 
    p. 14.}
    \item Defamation doesn't protect ``injury to the 
    feelings.''\footnote{Casebook p. 15.}
    \item Intellectual property protections---for instance, the right to 
    prevent publication---are part of a broader common law right to 
    privacy.\footnote{Casebook pp. 15--16.}
    \item There is a ``general right of the individual right of the individual 
    to be let alone.''\footnote{Casebook p. 18.} Invasion of privacy is a 
    common law tort.\footnote{Casebook pp. 18-20.}
    \item Protections don't apply to public figures (at least, not to public 
    officials---protections \emph{do} apply to ``modest and retiring 
    individuals,'' so an involuntary public figure like Sidis would probably 
    have a cause of action).\footnote{Casebook p. 21.}
\end{enumerate}

\paragraph{Four Privacy Torts: Prosser, \emph{Privacy}}

\begin{enumerate}
    \item Intrusion.\footnote{Casebook p. 27.}
    \item Disclosure.
    \item False light.
    \item Appropriation.
\end{enumerate}

\paragraph{Adopting the Prosser Torts: \emph{Lake v. Wal-Mart Stores, Inc.}}

\begin{enumerate}
    \item Joining most other states, Minnesota adopted the Prosser privacy 
    torts (except false light, because it is too close to defamation and 
    because it raises First Amendment concerns).\footnote{Casebook pp. 29--31.}
\end{enumerate}

\paragraph{Privacy and Other Areas of Law}

\begin{enumerate}
    \item Torts: Prosser's four, breach of confidentiality, defamation, 
    infliction of emotional distress.
    \item Evidence: privileged relationships.
    \item Property: trespass. Also, should we treat personal information as 
    property?
    \item Contract: private agreements.
    \item Criminal law: injury (to body and property), trespass, 
    stalking/harassing, blackmail, wiretapping, identity theft.
\end{enumerate}

\subsubsection{Constitutional Law}

\begin{enumerate}
    \item Federal: First Amendment (anonymous speech), Third (privacy of the 
    home), Fourth (many interpretations), Fifth (privilege against 
    self-incrimination). \emph{Griswold} (marital privacy), \emph{Whalen} 
    (``constitutional right to information privacy'').
    \item State: many state constitutions (e.g., California) have explicit 
    privacy protections.
\end{enumerate}

\subsubsection{Statutory Law}

\begin{enumerate}
    \item Federal: many specific statutes.\footnote{See casebook pp. 36--39.} 
    Also, the general Privacy Act of 1974.
    \item State: many specific statutes, but less than a third have enacted 
    ``omnibus data protection laws.''\footnote{Casebook p. 39.}
\end{enumerate}

\subsubsection{International Law}

\begin{enumerate}
    \item OECD guidelines, APEC Privacy Framework.\footnote{Casebook pp, 
    39--40.}
\end{enumerate}

\subsection{Perspectives on Privacy}

\subsubsection{Philosophy}

\paragraph{The Concept of Privacy and the Right to Privacy}

\begin{enumerate}
    \item The \emph{concept} of privacy is distinct from the \emph{right} to 
    privacy.
\end{enumerate}

\paragraph{The Public and Private Spheres}

\begin{enumerate}
    \item Arendt, Mill.\footnote{Casebook pp. 40--41.}.
\end{enumerate}

\subsubsection{Definition and Value}

\paragraph{Westin, \emph{Privacy and Freedom}}

\begin{enumerate}
    \item Surveillance is necessary in order to enforce social norms.
    \item Four states of privacy: solitude, intimacy, anonymity, reserve.
    \item Functions of privacy: personal autonomy, self evaluation, limited 
    and protected communication.\footnote{Casebook pp. 42--45.}
\end{enumerate}

\paragraph{Cohen, \emph{Examined Lives: Informational Privacy and the Subject 
as Object}}

\begin{enumerate}
    \item Autonomy requires a zone of insulation from scrutiny and 
    interference.
    \item ``~.~.~.~the experience of being watched will constrain, ex ante, 
    the acceptable spectrum of belief and behavior.''\footnote{Casebook pp. 
    48-49.}
\end{enumerate}

\paragraph{Solove, \emph{Conceptualizing Property}}

\begin{enumerate}
    \item ``When we state that we are protecting ``privacy,'' we are claiming 
    to guard against disruptions to certain practices.''\footnote{Casebook p. 
    52.}
    \item Privacy depends on context; there is no common 
    denominator.\footnote{Casebook p. 53.}
    \item Reductionists: privacy can be reduced to other concepts and 
    rights.\footnote{Casebook p. 54.}
\end{enumerate}

\paragraph{Allen, \emph{Coercing Privacy}}

\begin{enumerate}
    \item Privacy is a ``foundation, a precondition of a liberal egalitarian 
    society.''\footnote{Casebook p. 55.} So we should sometimes force it on 
    people---for instance, through public nudity laws.
\end{enumerate}

\paragraph{Schwartz, \emph{Privacy and Democracy in Cyberspace}}

\begin{enumerate}
    \item Control over information is a flawed understanding of privacy in 
    digital contexts. For instance, it assumes we are autonomous, but we often 
    are not---for instance, if we accept a boilerplate EULA that allows the 
    company to do anything with our information.\footnote{Casebook p. 57.}
\end{enumerate}

\paragraph{Simitis, \emph{Reviewing Privacy in an Information Society}}

\begin{enumerate}
    \item Personal information can enforce standards of behavior, which means 
    that increased surveillance can facilitate ``adjustment'' but that it may 
    be harmful to democracy.\footnote{Casebook pp. 59--61.}
\end{enumerate}

\subsubsection{Critics}

\paragraph{Posner, \emph{The Right of Privacy}}

\begin{enumerate}
    \item Gossip can inform. Many people present themselves deceptively, so 
    nosiness can be helpful.\footnote{Casebook pp. 62--63.}
\end{enumerate}

\paragraph{Cate, \emph{Principles of Internet Privacy}}

\begin{enumerate}
    \item U.S. law historically has a strong preference for the free flow of 
    information, which has ``significant economic and social benefits'' (e.g., 
    price signals).\footnote{Casebook pp. 66--68.}
    \item 
\end{enumerate}

\subsubsection{Feminism and Privacy}

\paragraph{Privacy as Gender Oppression: \emph{State v. Rhodes}}

\begin{enumerate}
    \item Should a husband be convicted for whipping his wife? No---family 
    privacy outweighs the need to punish impulsive violence.
\end{enumerate}

\paragraph{Siegel, \emph{``The Rule of Love'': Wife Beating as Prerogative and 
Privacy}}

\begin{enumerate}
    \item \emph{Rhodes} is one of many cases in which privacy serves as a tool 
    of gender oppression.\footnote{Casebook pp. 72--74.}
\end{enumerate} 

\paragraph{MacKinnon, \emph{Toward a Feminist Theory of the State}}

\begin{enumerate}
    \item Privacy can perpetuate subordination.\footnote{Casebook pp. 74--75.}
\end{enumerate}

\paragraph{Allen, \emph{Uneasy Access: Privacy for Women in a Free Society}}

\begin{enumerate}
    \item MacKinnon goes too far. Privacy is not an inherent threat to 
    women. We should seek ``adequate and meaningful 
    privacy~.~.~.~''\footnote{Casebook pp. 75--76.}
\end{enumerate}
