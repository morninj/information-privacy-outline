\section{International Privacy Law}

\subsection{OECD Guidelines}

\begin{enumerate}
    \item U.S. is a member.\footnote{Casebook p. 1063--65 ff.}
    \item Enacted 1980. Grew out of a need for greater interoperability among 
    international privacy frameworks.
    \item Guidelines are \textbf{not binding anywhere}.
    \item Guidelines:
    \begin{enumerate}
        \item Collection limitation.
        \item Data quality.
        \item Purpose specification.
        \item Use limitation.
        \item Security safeguards.
        \item Openness/transparency.
        \item Individual participation.
        \item Accountability.
    \end{enumerate}
    \item 2013: OECD adopts a revised recommendation.
    \item Accountability requires a ``privacy management program.''
    \item Notification is required for significant data breaches that are 
    likely to have adverse effects.
\end{enumerate}

\subsection{Privacy Protection in Europe}

\subsubsection{Whitman, \emph{The Two Western Cultures of Privacy: Dignity vs. 
Liberty}}

\begin{enumerate}
    \item United States: freedom from governmental intrusion; personal 
    sovereignty---e.g., no national ID system.\footnote{Casebook p. 1065--70.}
    \item Europe: personal dignity, respect, honor, face-saving---e.g., no 
    intrusive credit reports.
\end{enumerate}

\subsubsection{European Convention on Human Rights Article 8}

\begin{itemize}
    \item EU: PII is all information \textbf{identifiable} to a person.
    \item ``Everyone has the right to respect for his private and family life, 
    his home, and his correspondence.''
    \item % TODO 1065-70
    \item Paul Schwartz: part of human dignity is being undignified.
\end{itemize}

\paragraph{Privacy and the Media: \emph{Von Hannover v. Germany}} % TODO 
clarify heading

\begin{enumerate}
    \item Distinction between reporting facts and reporting intimate details.
    \item Publication does not contribute to any debate of general interest to 
    society.
    \item % TODO 1073-83
\end{enumerate}

\paragraph{Privacy and the Media: \emph{Mosley v. The United Kingdom}} % TODO 
clarify heading

\begin{enumerate}
    \item ``~.~.~.~applicant was hardly exaggerating when he said that his life 
    was ruined.''\footnote{Casebook p. 1084.}
    \item States have a ``wide margin of appreciation'' in deciding how to 
    ``respect'' Article 8.
    \item Mosley sought press pre-notification requirement. The court rejected 
    his request because Article 8 is about letting states adopt their own 
    balance between privacy protections and freedom of expression.
    \item Follow-up: a lower French court ordered Google to remove the Mosley 
    photos.
    \item % TODO 1083-94
\end{enumerate}

\subsubsection{The European Union Data Protection Directive}

\begin{enumerate}
    \item % TODO 1109-18
\end{enumerate}

\paragraph{\emph{Criminal Proceedings against Bodil Lindqvist}}

\begin{enumerate}
    \item % TODO 1118-23
\end{enumerate}

\paragraph{Article 8 and Harmonization}

\begin{enumerate}
    \item % TODO 1123-34
\end{enumerate}

\paragraph{Supervisory Authority and Individual Remedies}

\begin{enumerate}
    \item % TODO 1124-27
\end{enumerate}

% TODO schwartz, eu privacy and the cloud
